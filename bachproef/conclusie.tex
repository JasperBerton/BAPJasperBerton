%%=============================================================================
%% Conclusie
%%=============================================================================

\chapter{Conclusie}%
\label{ch:conclusie}

% TODO: Trek een duidelijke conclusie, in de vorm van een antwoord op de
% onderzoeksvra(a)g(en). Wat was jouw bijdrage aan het onderzoeksdomein en
% hoe biedt dit meerwaarde aan het vakgebied/doelgroep? 
% Reflecteer kritisch over het resultaat. In Engelse teksten wordt deze sectie
% ``Discussion'' genoemd. Had je deze uitkomst verwacht? Zijn er zaken die nog
% niet duidelijk zijn?
% Heeft het onderzoek geleid tot nieuwe vragen die uitnodigen tot verder 
%onderzoek?

Dit onderzoek stemt voort uit volgende onderzoeksvraag:
\begin{itemize}
    \item Hoe kan het ontwikkelen van een interne applicatie het handmatig testen van een privaat netwerk vereenvoudigen?
\end{itemize}

Na een grondige literatuurstudie en een daaropvolgend praktisch onderzoek is het mogelijk om hierop een antwoord te formuleren. De verkregen Proof of Concept en het gebruik hiervan legt namelijk duidelijke pijn- en pluspunten te boven. Deze applicatie werd ontwikkeld om het proces omtrent de testen toegepast op een netwerk te stroomlijnen. Voor de arbeidsintensieve taak omtrent het veld zelf afgaan is dit namelijk niet mogelijk. Een persoon die  een bepaalde tour rond het terrein loopt zal altijd noodzakelijk blijven. Wel kon er opgemaakt worden dat de data die verzameld werd door de POC adequaat van niveau was ten opzichte van het commercieel alternatief. De data verwerken die voortkwam uit beide applicaties bleek wel minder tijd consumerend bij de POC. Dit vanwege het feit dat deze werd bijgehouden in een externe database ten opzichte van een data dump in de vorm van een .txt bestand op het apparaat zelf bij de andere applicatie. Nadien is er dus minder werk om tot conclusies te komen gebruikmakende van de aanwezige data. Er was in de POC ook meer flexibiliteit aanwezig omtrent de soorten testen die men kon uitvoeren. Zo was het mogelijk om een speciale routine te voorzien gebaseerd op een specifieke situatie i.p.v. de één soort test die de andere applicatie aanbood.\\

Uit dit onderzoek kunnen wel andere onderzoeksvragen geformuleerd worden namelijk:
\begin{itemize}
    \item In welke capaciteit heeft het direct verzenden van data een invloed op het netwerk dat men test?
    \item Kan een applicatie testen op de achtergrond uitvoeren en resulteert dit in gelijkaardige resultaten ten opzichte van testen uitgevoerd op de voorgrond?
\end{itemize}

