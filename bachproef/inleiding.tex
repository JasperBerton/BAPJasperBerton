%%=============================================================================
%% Inleiding
%%=============================================================================

\chapter{\IfLanguageName{dutch}{Inleiding}{Introduction}}%
\label{ch:inleiding}

Doorheen de afgelopen decennia is de mens steeds afhankelijker geworden van een bepaald apparaat namelijk de gsm. Er zijn al verscheidene applicaties die ons het leven gemakkelijker maken maar ze leggen wel een ander pijnpunt bloot, namelijk de nood om continu verbonden zijn met een netwerk. De kracht van openbare netwerken groeit met de dag maar in verscheidene situaties is er nog steeds nood voor extra ondersteuning zodat men in situaties waar het nodig is niet zonder een verbinding valt. \\

Een mogelijke oplossing om te zorgen dat cruciale technologie blijft draaien kan men gebruik maken van een privaat netwerk. Hierbij kan de consument zelf beslissen wie er toegang heeft tot het netwerk om zo tegen te gaan dat er een mogelijke overbelasting is. Een nadeel dat echter hierbij opkomt is dat men dan wel verwacht dat dit netwerk ten alle tijde in werking blijft. Volgens \textcite{Dux2023} was er een marktpotentieel van 1212 bedrijven met implementaties van een privaat netwerk aanwezig. 
\section{\IfLanguageName{dutch}{Probleemstelling}{Problem Statement}}%
\label{sec:probleemstelling}

%%%Uit je probleemstelling moet duidelijk zijn dat je onderzoek een meerwaarde heeft voor een concrete doelgroep. De doelgroep moet goed gedefinieerd en afgelijnd zijn. Doelgroepen als ``bedrijven,'' ``KMO's'', systeembeheerders, enz.~zijn nog te vaag. Als je een lijstje kan maken van de personen/organisaties die een meerwaarde zullen vinden in deze bachelorproef (dit is eigenlijk je steekproefkader), dan is dat een indicatie dat de doelgroep goed gedefinieerd is. Dit kan een enkel bedrijf zijn of zelfs één persoon (je co-promotor/opdrachtgever).

Dit onderzoek richt er zich op om een meerwaarde te bieden voor Citymesh. Dit is een bedrijf dat zich vooral toespitst op business to business connectiviteitsoplossingen. Producten die ze aanbieden bevatten maar beperken zich niet tot permanente en tijdelijke private netwerken, drone surveillance alsook onderzoek naar opkomende technologieën omtrent cellulaire netwerken. Dit bedrijf is opgedeeld in verschillende groepen met elk hun taak. Naast de basis onderverdelingen zoals HR en Legal is er ook nog een bepaalde sectie genaamd Flex. Zij spitsen zich toe op het opzetten van tijdelijke netwerken voor festivals of evenementen bijvoorbeeld Graspop Metal Meeting. Deze opstellingen zijn dan vaak om er voor te zorgen dat de ruggengraat van de ondersteunde operatie blijft draaien zonder een invloed te ondervinden van de drukte en last op het publiek netwerk. Wat men hieronder kan verstaan is een mogelijkheid voor de betaalterminals om te verbinden met het internet, telefonie of zelfs communicatie via walkietalkies. Dit team zou baten bij een gestroomlijnde oplossing omtrent het testen van een netwerk zodat men versneld de kwaliteit van de opstelling kan verzekeren gebaseerd op verzamelde data.

\section{\IfLanguageName{dutch}{Onderzoeksvraag}{Research question}}%
\label{sec:onderzoeksvraag}

%%Wees zo concreet mogelijk bij het formuleren van je onderzoeksvraag. Een onderzoeksvraag is trouwens iets waar nog niemand op dit moment een antwoord heeft (voor zover je kan nagaan). Het opzoeken van bestaande informatie (bv. ``welke tools bestaan er voor deze toepassing?'') is dus geen onderzoeksvraag. Je kan de onderzoeksvraag verder specifiëren in deelvragen. Bv.~als je onderzoek gaat over performantiemetingen, dan 


\section{\IfLanguageName{dutch}{Onderzoeksdoelstelling}{Research objective}}%
\label{sec:onderzoeksdoelstelling}

%%Wat is het beoogde resultaat van je bachelorproef? Wat zijn de criteria voor succes? Beschrijf die zo concreet mogelijk. Gaat het bv.\ om een proof-of-concept, een prototype, een verslag met aanbevelingen, een vergelijkende studie, enz.

De uiteindelijke uitkomst die verwacht word is een proof-of-concept in de vorm van een android applicatie. Deze applicatie zou dan getest worden tegenover een commercieel alternatief dat momenteel in gebruik is door de vragende partij naar dit alternatief. De applicatie zou als basis functionaliteit de mogelijkheid moeten geven aan de eindgebruiker om het netwerk waar mee men verbonden is te onderwerpen aan basis connectiviteitstesten. Hieruit kunnen dan conclusies genomen worden om gewogen beslissingen te nemen naar de toekomst.

\section{\IfLanguageName{dutch}{Opzet van deze bachelorproef}{Structure of this bachelor thesis}}%
\label{sec:opzet-bachelorproef}

% Het is gebruikelijk aan het einde van de inleiding een overzicht te
% geven van de opbouw van de rest van de tekst. Deze sectie bevat al een aanzet
% die je kan aanvullen/aanpassen in functie van je eigen tekst.

De rest van deze bachelorproef is als volgt opgebouwd:

In Hoofdstuk~\ref{ch:stand-van-zaken} wordt een overzicht gegeven van de stand van zaken binnen het onderzoeksdomein, op basis van een literatuurstudie.

In Hoofdstuk~\ref{ch:methodologie} wordt de methodologie toegelicht en worden de gebruikte onderzoekstechnieken besproken om een antwoord te kunnen formuleren op de onderzoeksvragen.

% TODO: Vul hier aan voor je eigen hoofstukken, één of twee zinnen per hoofdstuk

In Hoofdstuk~\ref{ch:conclusie}, tenslotte, wordt de conclusie gegeven en een antwoord geformuleerd op de onderzoeksvragen. Daarbij wordt ook een aanzet gegeven voor toekomstig onderzoek binnen dit domein.