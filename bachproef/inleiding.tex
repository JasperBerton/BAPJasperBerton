%%=============================================================================
%% Inleiding
%%=============================================================================

\chapter{\IfLanguageName{dutch}{Inleiding}{Introduction}}%
\label{ch:inleiding}

Doorheen de afgelopen decennia is de mens steeds afhankelijker geworden van een zeker apparaat, namelijk de gsm. Er zijn al applicaties die ons het leven gemakkelijker maken maar ze leggen wel een ander pijnpunt bloot , namelijk de nood om continu verbonden te zijn met het wereldwijde web. De kracht van openbare netwerken en Wifi installaties groeien met de dag maar in verscheidene situaties is er nog steeds nood voor extra ondersteuning zodat in bepaalde omstandigheden indien het nodig is iemand niet zonder een verbinding valt. \\
Een privaat netwerk kan een oplossing bieden om te zorgen dat noodzakelijke technologie blijft draaien. Hierbij kan de beheerder hiervan zelf beslissen wie er toegang heeft tot de infrastructuur om zo tegen te gaan dat er een overbelasting plaatsvind. Men mag dit wel niet verwarren met een Wifi netwerk dat via een wachtwoord is beveiligd want de data transporteert zich via een spectrum van binnen een cellulair netwerk. Doordat de aanbieder eigenaar is van dit spectrum is deze bandbreedte volledig vrij voor de consument. Een nadeel dat hierbij opkomt is dat men dan wel verwacht dat deze opstelling te alle tijde in werking blijft. Men kan dan namelijk wel de vraag stellen of er netwerken genoeg zijn om specifieke test applicaties te verantwoorden. Volgens \textcite{Dux2023}  was er een marktpotentieel van 1212 bedrijven met toepassingen van een privaat netwerk aanwezig. Om tot dit cijfer te komen werd door de Global mobile Suppliers Association rondvraag gedaan bij de bedrijven die het materiaal voorzien voor de opzet van dit type netwerken. Bedrijven die hiervoor informatie verstreken waren Ericsson, Keysight Technologies, Airspan, Mavenir, Nokia \& Onelayer. Uit dit onderzoek bleek ook dat dit een stijging is in vergelijking met de voorgaande jaren en dat steeds meer sectoren een opportuniteit zien in het bezitten van hun eigen spectrum. Daardoor kan er een voorzichtige speculatie geformuleerd worden dat dit aantal de komende jaren nog zal stijgen.
\section{\IfLanguageName{dutch}{Probleemstelling}{Problem Statement}}%
\label{sec:probleemstelling}

%%%Uit je probleemstelling moet duidelijk zijn dat je onderzoek een meerwaarde heeft voor een concrete doelgroep. De doelgroep moet goed gedefinieerd en afgelijnd zijn. Doelgroepen als ``bedrijven,'' ``KMO's'', systeembeheerders, enz.~zijn nog te vaag. Als je een lijstje kan maken van de personen/organisaties die een meerwaarde zullen vinden in deze bachelorproef (dit is eigenlijk je steekproefkader), dan is dat een indicatie dat de doelgroep goed gedefinieerd is. Dit kan een enkel bedrijf zijn of zelfs één persoon (je co-promotor/opdrachtgever).

Dit onderzoek richt er zich op om een meerwaarde te bieden voor Citymesh. Dit is een bedrijf dat zich toespitst op business to business connectiviteitsoplossingen. Producten die ze aanbieden bevatten maar beperken zich niet tot permanente en tijdelijke private netwerken, drone surveillance en ook onderzoek naar opkomende technologieën over cellulaire netwerken. Dit bedrijf is opgedeeld in groepen met elk hun taak. Naast de basis onderverdelingen zoals Human Resources en Legal is er ook nog een zekere sectie genaamd Flex. Zij spitsen zich toe op het opzetten van tijdelijke netwerken voor festivals of evenementen bijvoorbeeld Graspop Metal Meeting. Deze opstellingen dienen als een ruggengraat zodat de ondersteunde operatie blijft draaien zonder een invloed te ondervinden van de drukte en last die dit genereert op het publiek netwerk. Een voorbeeld van wat hiermee bedoeld wordt is een mogelijkheid voor de betaalterminals om te verbinden met het internet maar ook telefonie of zelfs communicatie via walkietalkies. Dit team zou baten bij een gestroomlijnde oplossing omtrent het testen van een netwerk zodat men versneld de kwaliteit van een opstelling kan verzekeren gebaseerd op verzamelde data.

\section{\IfLanguageName{dutch}{Onderzoeksvraag}{Research question}}%
\label{sec:onderzoeksvraag}

%%Wees zo concreet mogelijk bij het formuleren van je onderzoeksvraag. Een onderzoeksvraag is trouwens iets waar nog niemand op dit moment een antwoord heeft (voor zover je kan nagaan). Het opzoeken van bestaande informatie (bv. ``welke tools bestaan er voor deze toepassing?'') is dus geen onderzoeksvraag. Je kan de onderzoeksvraag verder specifiëren in deelvragen. Bv.~als je onderzoek gaat over performantiemetingen, dan 

Dit onderzoek spitst zich toe op de ontwikkeling van een applicatie omtrent het testen van private netwerken ten opzichte van commercieel beschikbare applicaties om een privaat netwerk te testen. Specifiek voor het verbeteren van een zeer handmatig en tijdsintensief proces om de kwaliteit van een netwerk vast te stellen. Dit alles resulteert in volgende onderzoeksvraag:
\begin{itemize}
    \item Hoe kan het ontwikkelen van een interne applicatie het handmatig testen van een privaat netwerk vereenvoudigen?
\end{itemize}

De Proof-of-Concept zal bestaan uit een Android applicatie. Deze heeft als doel om zoveel mogelijk informatie omtrent de werking van het cellulair netwerk waarmee het verbonden is te verzamelen. Deze informatie bevat componenten zoals bijvoorbeeld de latentie, locatie van het apparaat, welk netwerk men mee verbonden is. Al deze data kan dan ingezet worden om een conclusie te maken omtrent de werking van het netwerk. \\

Een antwoord op volgende deelvragen helpt met eerder vernoemde onderzoeksvraag te beantwoorden:

\begin{itemize}
    \item welke tijdswinst kan men mogelijk halen door middel van de proof-of-concept tegenover de huidige commercieel beschikbare applicatie bij het testen van eenzelfde netwerk?
    \item Welke data kan men ophalen omtrent de interacties tussen smartphone en telefoonmast via de door Android aangereikte eindpunten?
    \item Welke functionaliteiten ontbreken in de commercieel beschikare applicaties die wel bruikbaar kunnen zijn bij het testen van een netwerk?
\end{itemize}

\section{\IfLanguageName{dutch}{Onderzoeksdoelstelling}{Research objective}}%
\label{sec:onderzoeksdoelstelling}

%%Wat is het beoogde resultaat van je bachelorproef? Wat zijn de criteria voor succes? Beschrijf die zo concreet mogelijk. Gaat het bv.\ om een proof-of-concept, een prototype, een verslag met aanbevelingen, een vergelijkende studie, enz.

De uiteindelijke uitkomst die verwacht wordt is een proof-of-concept in de vorm van een android applicatie. Deze app zou dan getest worden tegenover een commercieel alternatief dat momenteel in gebruik is door de het Flex team binnen Citymesh. De applicatie zou als basisfunctionaliteit de mogelijkheid moeten bieden aan de eindgebruiker om het netwerk waar men mee verbonden is te onderwerpen aan een reeks connectiviteitstesten. Hieruit kunnen dan conclusies genomen worden om beslissingen te nemen naar de toekomst omtrent aanpassingen aan het netwerk indien nodig.

\section{\IfLanguageName{dutch}{Opzet van deze bachelorproef}{Structure of this bachelor thesis}}%
\label{sec:opzet-bachelorproef}

% Het is gebruikelijk aan het einde van de inleiding een overzicht te
% geven van de opbouw van de rest van de tekst. Deze sectie bevat al een aanzet
% die je kan aanvullen/aanpassen in functie van je eigen tekst.

De rest van deze bachelorproef is als volgt opgebouwd:

In Hoofdstuk~\ref{ch:stand-van-zaken} wordt een overzicht gegeven van de stand van zaken binnen het onderzoeksdomein, op basis van een literatuurstudie.

In Hoofdstuk~\ref{ch:methodologie} wordt de methodologie toegelicht en worden de gebruikte onderzoekstechnieken besproken om een antwoord te kunnen formuleren op de onderzoeksvragen.

In Hoofdstuk~\ref{ch:proofofconcept} wordt er een diepere kijk gegeven in de werking en opbouw van de POC die doorheen deze bachelorproef is opgebouwd alsook de verkregen data van beide applicaties word toegelicht.

In Hoofdstuk~\ref{ch:conclusie}, tenslotte, wordt de conclusie gegeven en een antwoord geformuleerd op de onderzoeksvragen. Daarbij wordt ook een aanzet gegeven voor toekomstig onderzoek binnen dit domein.