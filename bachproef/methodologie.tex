%%=============================================================================
%% Methodologie
%%=============================================================================

\chapter{\IfLanguageName{dutch}{Methodologie}{Methodology}}%
\label{ch:methodologie}

%% TODO: In dit hoofstuk geef je een korte toelichting over hoe je te werk bent
%% gegaan. Verdeel je onderzoek in grote fasen, en licht in elke fase toe wat
%% de doelstelling was, welke deliverables daar uit gekomen zijn, en welke
%% onderzoeksmethoden je daarbij toegepast hebt. Verantwoord waarom je
%% op deze manier te werk gegaan bent.
%% 
%% Voorbeelden van zulke fasen zijn: literatuurstudie, opstellen van een
%% requirements-analyse, opstellen long-list (bij vergelijkende studie),
%% selectie van geschikte tools (bij vergelijkende studie, "short-list"),
%% opzetten testopstelling/PoC, uitvoeren testen en verzamelen
%% van resultaten, analyse van resultaten, ...
%%
%% !!!!! LET OP !!!!!
%%
%% Het is uitdrukkelijk NIET de bedoeling dat je het grootste deel van de corpus
%% van je bachelorproef in dit hoofstuk verwerkt! Dit hoofdstuk is eerder een
%% kort overzicht van je plan van aanpak.
%%
%% Maak voor elke fase (behalve het literatuuronderzoek) een NIEUW HOOFDSTUK aan
%% en geef het een gepaste titel.

Gedurende de eerste fase van het onderzoek zal een literatuurstudie opgesteld worden. Deze is terug te vinden in hoofdstuk 2 onder de stand van zaken en bespreekt onderwerpen gaande van de geschiedenis omtrent telecommunicatie netwerken tot de structuur van een LTE netwerk en zelfs een diepgaande uitleg omtrent verschillende termen die vaak voorkomend zijn wanneer men de kwaliteit van een bepaald signaal bespreekt. \\

De tweede fase bestaat er uit om een requirement analyse uit te voeren zodat er een beeld kan gevormd worden aan wat de applicatie dient te voldoen om effectief ingezet te kunnen worden in het werkveld om zo functioneel netwerken te testen. De voorwaarden die bij deze stap neergezet worden kan men dan nadien ook inzetten om de nieuwe applicatie te vergelijken met andere applicaties die al reeds op de markt zijn, of in gebruik zijn binnen Citymesh. Dit onderdeel zal hieronder in een apart hoofdstuk besproken worden. \\

Doorheen de derde fase zal er dan een Proof-of-Concept opgesteld worden gebaseerd op de eerder gestelde voorwaarden. Dit zal gebeuren in meerdere iteraties en met terugkoppeling naar de poule van eindgebruikers voor de te ontwikkelen tool. \\

Ten slotte worden de oude en nieuwe tool tegenover elkaar gezet in een labo omgeving waarin testen zullen uitgevoerd worden om zo te zien of er significante voordelen zijn geboekt bij het opbouwen van de nieuwe applicatie ten opzichte van de reeds bestaande technologie. Hierover zal dan een conclusie gevormd worden met advies richting Citymesh zodat men gebruik makende van de beste tool verder hun netwerken kan testen.

\section{Requirementsanalyse}

Om een grondig antwoord te formuleren op de gestelde vraag is het noodzakelijk om vast te leggen wat er van een bepaalde applicatie met dit doel verwacht word. Om deze functionaliteiten een bepaalde prioriteit toe te kennen zal gebruik gemaakt worden van de MoScoW-methode. Hierbij zal elke nood onderverdeeld een plaats krijgen in een bepaalde categorie. Eest en vooral is er de 'Must have', dit bestaat uit de functionaliteiten die een tool moet hebben of deze kan niet gerekend worden als een mogelijke oplossing voor het probleem. Een trap lager is er dan de 'Should have', dit representeert het deel dat belangrijk is, maar niet essentieel. Ten derde is er dan de 'Could have' categorie, dit is dan de groep van functionaliteiten met een lage prioriteit maar waarvan het bestaan wel goed is voor het product. Ten slotte is er ook nog de 'Won't have' functionaliteiten. Hieronder verstaat men de prioriteiten waaraan geen aandacht moet besteed worden vanwege het feit dat dit een lage prioriteit heeft of het niet nodig is in de huidige iteratie richting een product. \autocite{Khan2015} De informatie die deze lijst bevat werd vastgelegd aan de hand van gesprekken met verscheidene doelgroepen waaronder het Flex en Research team binnen Citymesh.

\begin{itemize}
    \item \textbf{Must have}
    \begin{itemize}
        \item Locatie bepalen van een eindgebruiker.
        \item Download \& Upload limiet van netwerk meten.
        \item Latentie tussen UE en bepaald IP adres vastleggen.
        \item Informatie bijhouden in een database voor verder onderzoek.
        \item Visualisatie van desbetreffende data op een administratief portaal.
        \item Indicatie naar eindgebruiker wat er gebeurt gedurende een bepaalde test die loopt.
    \end{itemize}
    \item \textbf{Should have}
    \begin{itemize}
        \item detecteren van extra informatie omtrent netwerk zoals bijvoorbeeld RSSI, RSSQ, id van de cell waarmee UE verbonden is etc.
        \item Opbreken van informatie gebaseerd voor welk project het nodig is.
        \item Robuuste navigatie tussen verscheidene componenten
    \end{itemize}
    \item \textbf{Could have}
    \begin{itemize}
        \item Testen uitvoeren op de achtergrond van een UE.
        \item Van op afstand een test activeren.
        \item Aanduiden van belangrijke locaties op een kaart component.
        \item Genereren van rapporten gebaseerd op gemeten data.
    \end{itemize}
    \item \textbf{Won't have}
    \begin{itemize}
        \item Autorisatie van gebruiker die instrument bedient.
    \end{itemize}
\end{itemize}

\section{Beoordelingsschaal}

Wanneer de commerciële applicatie en de proof-of-concept zich meten aan de voorwaarden die hierboven gesteld staan, ontwikkeld er zich de nood voor een bepaalde schaal om deze een score toe te kunnen kennen. Om deze problematiek op te lossen zal gebruik gemaakt worden van een puntenschaal gaande van 1 tot 5. In het vervolg van dit hoofdstuk volgt een toelichting voor wat elke score betekend voor een specifieke requirement. Sommige van voorgaande items zullen niet herhaald worden in deze categorie vanwege het feit dat sommige categorieën nominaal zijn in plaats van ordinaal.

\subsection{Locatie bepalen van een eindgebruiker}

\begin{itemize}
    \item Score 1: De applicatie geeft geen mogelijkheid om de locatie van de gebruiker te achterhalen.
    \item Score 2: De applicatie heeft een mogelijkheid om de locatie te bepalen maar is zeer onnauwkeurig, maximaal tot op een tiental meter van de effectieve locatie.
    \item Score 3: Locatiebepaling is nauwkeurig tot op een meter, maar enkel buiten.
    \item Score 4: De tool kan de gebruiker zijn locatie tot op een meter nauwkeurig bepalen zowel binnen als buiten.
    \item Score 5: Locatie kan bepaald worden tot op een centimeter nauwkeurig, zowel binnen als buiten.
\end{itemize}

\subsection{Download \& Upload limiet van netwerk meten}

\begin{itemize}
    \item Score 1: Men kan beide snelheden niet achterhalen met de aangeboden tools in de applicatie.
    \item Score 2: Men kan maximaal 1 van beide snelheden meten.
    \item Score 3: Men kan beide waarden opmeten maar er zit een duidelijke vertraging op fluctuaties in het netwerk die worden opgemerkt, de ondergrens hiervoor is 30 seconden.
    \item Score 4: Veranderingen binnen het netwerk worden opgemerkt maar er zit een duidelijke vertraging tussen de verandering en het loggen hiervan, met als ondergrens 2 en bovengrens 30 seconden.
    \item Score 5: Fluctuaties in het netwerk worden binnen de seconde opgemerkt.
\end{itemize}

\subsection{Latentie tussen UE en bepaald IP adres vastleggen}

\begin{itemize}
    \item Score 1: Deze waarde kan niet bepaald worden.
    \item Score 2: Er is geen enkele parameter die mogelijks aangepast kan worden en de data die binnenkomt komt niet overeen met de effectief verwachte waarden.
    \item Score 3: De eindgebruiker heeft geen vrije keuze welk adres het wil testen, alsook de interval tijd kan niet aangepast worden maar de waarden zijn wel accuraat.
    \item Score 4: De eindgebruiker heeft geen vrije keuze welk adres het wil testen maar de waarden worden wel accuraat vastgelegd en de interval tussen elk verzonden pakket is ook een parameter die kan aangepast worden.
    \item Score 5: De eindgebruiker heeft vrije keuze welk adres het wil testen, alsook de intervaltijd is aanpasbaar en de waarden worden accuraat vastgelegd.
\end{itemize}

\subsection{Informatie bijhouden in een database voor verder onderzoek}

\begin{itemize}
    \item Score 1: Er is geen manier voor opslag van de data voorzien binnen de applicatie.
    \item Score 2: Er is een lokale database maar deze is extern niet bereikbaar en slaat data onregelmatig op.
    \item Score 3: De data bestaat in een lokale database en is consistent opgeslagen maar is extern niet beschikbaar.
    \item Score 4: De data bestaat in een databank, is consistent opgeslagen en is extern beschikbaar maar de lokale data vertoont onregelmatigheden omtrent de synchronisatie in verband met externe data.
    \item Score 5: De lokale databank bevat alle data, is consistent opgeslagen en is extern beschikbaar alsook de data is exact hetzelfde in vergelijking met de externe data. 
\end{itemize}

\subsection{Visualisatie van desbetreffende data op een administratief portaal}

\begin{itemize}
    \item Score 1: Er is geen administratief portaal aanwezig voor de bijhorende applicatie.
    \item Score 2: Er is een administratief portaal aanwezig maar de eindgebruiker moet data eigenhandig inladen.
    \item Score 3: Er is een administratief portaal aanwezig en de eindgebruiker hoeft de data niet zelf in te laden maar de opties voor data visualisatie zijn weinig tot niet bestaand.
    \item Score 4: Een portaal is aanwezig en de gebruiker hoeft niet zelf data in te laden, alsook is er een wijd gamma aan data visualisatie.
    \item Score 5: Een portaal is aanwezig met automatisch ingeladen data en er is een wijd gamma aan data visualisatie dat de gebruiker zelf kan aanpassen naar zijn/haar wensen.
\end{itemize}

\subsection{Indicatie naar eindgebruiker wat er gebeurt gedurende een bepaalde test die loopt}

\begin{itemize}
    \item Score 1: Er is geen indicatie richting gebruiker.
    \item Score 2: Gedurende een test is er een vorm van indicatie maar er is geen extra informatie omtrent duur of verloop van de test alsook geen melding indien iets misloopt.
    \item Score 3: Er is een vorm van indicatie zonder extra informatie maar wel met een melding indien iets misloopt.
    \item Score 4: Er is een vorm van indicatie met extra informatie omtrent de duur en met melding indien iets fout gaat.
    \item Score 5: Er is duidelijke indicatie met genoeg informatie richting eindgebruiker om eigen conclusies te nemen, alsook meldingen indien bepaalde processen fout lopen.
\end{itemize}

\subsection{Opbreken van informatie voor welk project het nodig is.}
\begin{itemize}
    \item Score 1: Er is geen mogelijkheid om een project parameter toe te voegen aan de data.
    \item Score 2: Er is een project parameter die toegevoegd is aan de data maar deze is hardcoded en kan niet aangepast worden door de gebruiker.
    \item Score 3: Er is een project parameter die toegevoegd is aan het data object die de gebruiker zelf kan kiezen, maar kan deze eenmalig instellen. 
    \item Score 4: Er is een project parameter die toegevoegd is aan het data object die de gebruiker zelf meerdere malen kan kiezen en aanpassen.
    \item Score 5: Er is een lijst beschikbaar met mogelijke project namen zodat de keuze van een gebruiker consistent blijft met de verwachte data die binnenkomt, dit word bijgehouden als een object dat meermaals kan aangepast worden.
\end{itemize}

\subsection{Versleuteling van data wegens sensitieve aard van gemeten waarde}

\begin{itemize}
    \item Score 1: Er is geen data die over een netwerk zich verplaatst.
    \item Score 2: Er is data die zich over een netwerk verplaatst maar deze is niet versleuteld.
    \item Score 3: De data is zwak geëncrypteerd en kan eenvoudig uitgelezen worden met behulp van beschikbare middelen.
    \item Score 4: De data is geëncrypteerd volgens een middelmatige standaard en kan deels ontcijferd worden met beschikbare tools.
    \item Score 5: De data is volgens een hoge standaard geëncrypteerd en kan niet ontcijferd worden met basis tools die ter beschikking zijn.  
\end{itemize}