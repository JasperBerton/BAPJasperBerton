%%=============================================================================
%% Methodologie
%%=============================================================================

\chapter{\IfLanguageName{dutch}{Methodologie}{Methodology}}%
\label{ch:methodologie}

%% TODO: In dit hoofstuk geef je een korte toelichting over hoe je te werk bent
%% gegaan. Verdeel je onderzoek in grote fasen, en licht in elke fase toe wat
%% de doelstelling was, welke deliverables daar uit gekomen zijn, en welke
%% onderzoeksmethoden je daarbij toegepast hebt. Verantwoord waarom je
%% op deze manier te werk gegaan bent.
%% 
%% Voorbeelden van zulke fasen zijn: literatuurstudie, opstellen van een
%% requirements-analyse, opstellen long-list (bij vergelijkende studie),
%% selectie van geschikte tools (bij vergelijkende studie, "short-list"),
%% opzetten testopstelling/PoC, uitvoeren testen en verzamelen
%% van resultaten, analyse van resultaten, ...
%%
%% !!!!! LET OP !!!!!
%%
%% Het is uitdrukkelijk NIET de bedoeling dat je het grootste deel van de corpus
%% van je bachelorproef in dit hoofstuk verwerkt! Dit hoofdstuk is eerder een
%% kort overzicht van je plan van aanpak.
%%
%% Maak voor elke fase (behalve het literatuuronderzoek) een NIEUW HOOFDSTUK aan
%% en geef het een gepaste titel.

Gedurende de eerste fase van het onderzoek zal een literatuurstudie opgesteld worden. Deze is terug te vinden in hoofdstuk 2 onder de stand van zaken en bespreekt onderwerpen gaande van de geschiedenis omtrent telecommunicatie netwerken tot de structuur van een LTE netwerk en zelfs een diepgaande uitleg omtrent verschillende termen die vaak voorkomend zijn wanneer men de kwaliteit van een bepaald signaal bespreekt. \\

De tweede fase bestaat er uit om een requirement analyse uit te voeren zodat er een beeld kan gevormd worden aan wat de applicatie dient te voldoen om effectief ingezet te kunnen worden in het werkveld om zo functioneel netwerken te testen. De voorwaarden die bij deze stap neergezet worden kan men dan nadien ook inzetten om de nieuwe applicatie te vergelijken met andere applicaties die al reeds op de markt zijn, of in gebruik zijn binnen Citymesh. Dit onderdeel zal hieronder in een apart hoofdstuk besproken worden. \\

Doorheen de derde fase zal er dan een Proof-of-Concept opgesteld worden gebaseerd op de eerder gestelde voorwaarden. Dit zal gebeuren in meerdere iteraties en met terugkoppeling naar de poule van eindgebruikers voor de te ontwikkelen tool. \\

Ten slotte worden de oude en nieuwe tool tegenover elkaar gezet in een labo omgeving waarin testen zullen uitgevoerd worden om zo te zien of er significante voordelen zijn geboekt bij het opbouwen van de nieuwe applicatie ten opzichte van de reeds bestaande technologie. Hierover zal dan een conclusie gevormd worden met advies richting Citymesh zodat men gebruik makende van de beste tool verder hun netwerken kan testen.

\section{Requirementsanalyse}

Om een grondig antwoord te formuleren op de gestelde vraag is het noodzakelijk om vast te leggen wat er van een bepaalde applicatie met dit doel verwacht word. Om deze functionaliteiten een bepaalde prioriteit toe te kennen zal gebruik gemaakt worden van de MoScoW-methode. Hierbij zal elke nood onderverdeeld een plaats krijgen in een bepaalde categorie. Eest en vooral is er de 'Must have', dit bestaat uit de functionaliteiten die een tool moet hebben of deze kan niet gerekend worden als een mogelijke oplossing voor het probleem. Een trap lager is er dan de 'Should have', dit representeert het deel dat belangrijk is, maar niet essentieel. Ten derde is er dan de 'Could have' categorie, dit is dan de groep van functionaliteiten met een lage prioriteit maar waarvan het bestaan wel goed is voor het product. Ten slotte is er ook nog de 'Won't have' functionaliteiten. Hieronder verstaat men de prioriteiten waaraan geen aandacht moet besteed worden vanwege het feit dat dit een lage prioriteit heeft of het niet nodig is in de huidige iteratie richting een product. \autocite{Khan2015}

