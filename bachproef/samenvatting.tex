%%=============================================================================
%% Samenvatting
%%=============================================================================

% TODO: De "abstract" of samenvatting is een kernachtige (~ 1 blz. voor een
% thesis) synthese van het document.
%
% Een goede abstract biedt een kernachtig antwoord op volgende vragen:
%
% 1. Waarover gaat de bachelorproef?
% 2. Waarom heb je er over geschreven?
% 3. Hoe heb je het onderzoek uitgevoerd?
% 4. Wat waren de resultaten? Wat blijkt uit je onderzoek?
% 5. Wat betekenen je resultaten? Wat is de relevantie voor het werkveld?
%
% Daarom bestaat een abstract uit volgende componenten:
%
% - inleiding + kaderen thema
% - probleemstelling
% - (centrale) onderzoeksvraag
% - onderzoeksdoelstelling
% - methodologie
% - resultaten (beperk tot de belangrijkste, relevant voor de onderzoeksvraag)
% - conclusies, aanbevelingen, beperkingen
%
% LET OP! Een samenvatting is GEEN voorwoord!

%%---------- Nederlandse samenvatting -----------------------------------------
%
%%---------- Samenvatting -----------------------------------------------------
% De samenvatting in de hoofdtaal van het document

\chapter*{\IfLanguageName{dutch}{Samenvatting}{Abstract}}

Testen van interne infrastructuur omtrent cellulaire verbindingen is cruciaal om zo een klant te kunnen verzekeren dat de beloofde kwaliteit is opgeleverd. Functionaliteit hiervoor bestaat al in een brede waaier voor de consument maar het aanbod op een commercieel niveau blijkt hier aan onder te doen. Vooral als men naar applicaties kijkt die vlot zijn in de omgang met sensitieve data en het vlug opsporen van problematiek binnen een netwerk. In een bedrijfscultuur kan ook geopperd worden dat elke seconde telt en dus alles dat vlugger kan zou dat ook moeten zijn. Hierdoor werd er vanuit Citymesh gevraagd om een mogelijke Proof of Concept uit te rollen om dit testen van een netwerk te optimaliseren. Deze POC werd dan uiteindelijk getest tegenover een alternatief dat reeds in het werkveld werd gebruikt op vlak van functionele en niet-functionele requirements. Uit dit onderzoek kon opgemaakt worden dat deze POC wel degelijk een tijdswinst boekt op het vlak van data visualisatie maar niet gedurende de manuele testfase ten opzichte van commerciële alternatieven. 

