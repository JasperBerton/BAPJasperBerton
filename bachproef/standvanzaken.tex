\chapter{\IfLanguageName{dutch}{Stand van zaken}{State of the art}}%
\label{ch:stand-van-zaken}

% Tip: Begin elk hoofdstuk met een paragraaf inleiding die beschrijft hoe
% dit hoofdstuk past binnen het geheel van de bachelorproef. Geef in het
% bijzonder aan wat de link is met het vorige en volgende hoofdstuk.

% Pas na deze inleidende paragraaf komt de eerste sectiehoofding.

Belangrijk voor men begint dieper in te gaan op welk cellulair netwerk zal getest worden is een geschiedenis van de verscheidene generaties van netwerken die er tot op heden zijn geweest. Door het verschil in te zien tussen de verschillende iteraties kan men inzien vanwaar de keuze voor een bepaalde technologie is genomen.

\section{Geschiedenis van het cellulair netwerk}

\subsection{Eerste generatie}

1G werd geïntroduceerd in de wereld in 1979. Deze technologie, eerst gebruikt door de inwoners van Japan, bracht de mogelijkheid om anderen te bellen maar het had ook een groot deel aan nadelen. Zo was een gesprek niet beveiligd, was roaming niet mogelijk en de geluidskwaliteit was zeer laag.
De reden dat een gesprek niet beveiligd is vanwege het feit dat er geen encryptie is over het netwerk dus eender wie die wou, kon meeluisteren. De trage downloadsnelheid die hier vernoemd werd bedroeg 2.4 kbps \autocite{Galazzo2020}. Neem nu bijvoorbeeld een korte film van 10 minuten in een mp4 formaat met codec H264 en een resolutie van 1920 x 1080p met een bitrate van 10.000 kbit/s. Zo'n bestand zou 775 MB in nemen. \autocite{Helme2019} De tijd dat het dus zou nemen om dit bestand te downloaden over het medium dat momenteel besproken word komt neer op 28 dagen, 22 uur, 26 minuten en 40 seconden. \autocite{Wooding2024}

\subsection{Tweede generatie}

2G maakte zijn introductie in de wereld gedurende het jaar 1991 in Finland op de rug van het Global System voor Mobile Communications of afgekort GSM. \autocite{Galazzo2020}. Deze volgende stap definieerde een significante stap naar verbetering ten opzichte van de eerste generatie. Men kon namelijk eindelijk berichten versturen naast bellen. Ook kon men multimedia bestanden delen zoals foto's en waren gesprekken vanaf dit punt wel geëncrypteerd en dus veilig van andermans oren. Doordat de populariteit van connectiviteit ook steeg was er ook een nood aan een efficiënter gebruik van het spectrum zodat meer apparaten het konden gebruiken. In tegenstelling tot 1G dat niet meer in wereldwijd gebruik is word 2G wel nog ingezet voor bepaalde toepassingen zoals The Internet of Things. \autocite{Henke2021} Indien men het voorbeeld van de kortfilm uit het eerste hoofdstuk wil herhalen nemen we de aangeboden waarde van een gemiddelde snelheid 0.2 Mbps. \autocite{Galazzo2020} Hierdoor zou de tijdsduur voor het downloaden van de kortfilm met identieke specificaties als in het eerste voorbeeld gelijkaardig zijn aan 8 uren en 20 seconden. \autocite{Wooding2024}

\subsection{Derde generatie}

3G werd voor het eerst ingezet in Japan op de eerste oktober in 2001. Dit kwam voor uit de steeds stijgende vraag van de wereldbevolking om geconnecteerd te zijn met elkaar en verscheidene acties te kunnen uitvoeren terwijl men onderweg was. Dit hield functionaliteiten in zoals videobellen, mails en video's verzenden aan de snelheid waarmee het toen mogelijk was om dit uit te voeren op een vaste computer. Om dit alles mogelijk te maken was er een nood aan hogere data snelheden en ook een beter gebruik van het spectrum om zo meer apparaten toe te laten op het netwerk. \autocite{Dulcey2020} Een technologie die hielp om dit allemaal mogelijk te maken was Code Division Multiple Access oftewel CDMA. Dit functioneert volgens een gespreid spectrum principe. Dit betekend dat de elektromagnetische energie wordt breder gespreid word zodat het kan gebruikt worden door een signaal met wijdere bandbreedte. Een nadeel van deze technologie is dat data en geluid niet gelijktijdig kunnen gespreid worden op het netwerk. Dit neveneffect zorgt er dan voor dat indien men aan het bellen is over CDMA dat er geen data binnengehaald kan worden zoals mails die binnenkomen. \autocite{Fendelman2021} Ten slotte om een eenvoudige vergelijking te maken met de andere cellulaire technologieën kan men een gemiddelde data transfer snelheid aannemen van 2 Mbps.\autocite{Galazzo2020} Om dit dan te vertalen naar de kortfilm bij voorgaande voorbeelden zou dit 50 minuten duren om deze data op te halen. \autocite{Wooding2024}

\subsection{Vierde generatie}

Door een nood aan hogere data snelheden werden in 2008 nieuwe standaarden gedefinieerd voor een verbeterde cellulaire technologie namelijk 4G. Men stelde als één van de doelen dat deze 1000 Mbit/s moest halen in data transfer snelheden. Een probleem waar men toen op stuitte was dat deze ambities te hoog waren gesteld. Hierdoor volgde een genoodzaakte beslissing in het tot leven roepen van LTE oftewel Long Term Evolution. Dit werd gezien als een tussenstap om zo wel een betere dienst aan te bieden tot men 4G volwaardig kon aanbieden. \autocite{Nicholls2022} Een techniek die toegepast werd in deze iteratie is Orthogonal Frequency Division Multiple Access (OFDMA). Dit werkt door subcarriers, de kleinste eenheid die data kan dragen, gegroepeerd worden in subkanalen. Men verdeelt deze eenheden dan in grotere groepen genaamd bursts die kunnen toegewezen worden aan bepaalde gebruikers. Deze manier van onderverdelen en toewijzen zorgt ervoor dat er een specifieke Quality of Service (QoS) en bandbreedte kan toegewezen worden gebaseerd op wie er verbonden is. Hierdoor word de beschikbare technologie beter ingezet dan bij vorige iteraties. \autocite{Friedmann2007}  



\pagebreak

Een cruciaal onderdeel wanneer men een bepaalde technologie wil testen is het begrijpen van de architectuur van desbetreffende technologie. Vandaar dat in het opkomende hoofdstuk zal besproken worden hoe een cellulair netwerk, meer specifiek dit van een long term evolution, of afgekort een LTE netwerk. Dit slaat dus specifiek op alle apparaten waarvan hun downloadsnelheid tussen de 100 megabits per seconde en de 1 gigabit per seconde ligt wanneer ze verbonden zijn met een cellulair netwerk. \autocite{Volle2024}

\section{Architectuur LTE netwerk}
Er kunnen 3 grote delen onderscheiden worden als er vanop een hoog niveau word neergekeken op de architectuur van deze technologie, namelijk de User Equipment oftewel UE, de Evolved UMTS Terrestrial Radio Access Network mogelijks afgekort tot E-UTRAN en ten slotte de Evolved Packet Core waarnaar verder gerefeerd zal worden als EPC. \autocite{Richard2021}
\\
Om een uitleg te vormen omtrent E-UTRAN moet er eerst begrepen worden wat een basis Radio Access Network inhoud. Dit onderdeel kan gezien worden als het radio gedeelte om ervoor te zorgen dat eindgebruikers kunnen communiceren met de rest van het systeem. Deze kan men in de omgeving spotten als palen ook wel nodeBs genoemd die drie componenten bevatten namelijk antennes, radio's en baseband units. Elektrische pulsen omzetten naar radio signalen is het doel van de antenne. Radio's daarentegen nemen de gereguleerde energie niveaus en gereserveerde frequentie in achting zodat digitale signalen kunnen omgezet worden in objecten die men draadloos kan versturen. Ten slotte zijn er dan nog de baseband units. Om draadloze communicatie namelijk mogelijk te maken is er een nood aan functies die logisch kunnen redeneren. Deze zijn in de laatst vernoemde component te vinden. \autocite{Jones2021}
\\
Om dan terug te komen naar wat de E-UTRAN dan specifiek inhoud kan er een vergelijking gemaakt worden met hoe het er in de voorgaande technologieën aan toe ging. Bij deze was er namelijk een nodeB zoals bij voorgaande besproken maar belangrijk is dat er hierbij ook een controller aanwezig is. De voornaamste taken van dit object waren het beheren van beschikbare middelen en het bijhouden van de verplaatsingen van een bepaalde eindgebruiker om zo deze door te kunnen geven van de ene nodeB naar de andere. Bij deze geëvolueerde versie daarvan valt de controller weg en smelt deze samen met de node zorgende voor een eNode. \autocite{Ghayas2019} Het feit dat alle functionaliteit verwoven zit in de node zelf zorgt voor lagere latency doordat er een hechte samenwerking is tussen de verschillende protocol lagen. Ook elimineert het een single point of failure en verlaagt het de kost. \autocite{Palat2011} \\

Nadat een UE verbonden is met een eNodeB word de data doorgestuurd naar de EPC. Dit systeem is een framework die verschillende functionaliteiten ondersteunt. Het verschil met voorgaande technologieën is dat het IP services zijn in plaats van circuit of packet switching. \autocite{Awati2024}




% Dit hoofdstuk bevat je literatuurstudie. De inhoud gaat verder op de inleiding, maar zal het onderwerp van de bachelorproef *diepgaand* uitspitten. De bedoeling is dat de lezer na lezing van dit hoofdstuk helemaal op de hoogte is van de huidige stand van zaken (state-of-the-art) in het onderzoeksdomein. Iemand die niet vertrouwd is met het onderwerp, weet nu voldoende om de rest van het verhaal te kunnen volgen, zonder dat die er nog andere informatie moet over opzoeken \autocite{Pollefliet2011}.

% Je verwijst bij elke bewering die je doet, vakterm die je introduceert, enz.\ naar je bronnen. In \LaTeX{} kan dat met het commando \texttt{$\backslash${textcite\{\}}} of \texttt{$\backslash${autocite\{\}}}. Als argument van het commando geef je de ``sleutel'' van een ``record'' in een bibliografische databank in het Bib\LaTeX{}-formaat (een tekstbestand). Als je expliciet naar de auteur verwijst in de zin (narratieve referentie), gebruik je \texttt{$\backslash${}textcite\{\}}. Soms is de auteursnaam niet expliciet een onderdeel van de zin, dan gebruik je \texttt{$\backslash${}autocite\{\}} (referentie tussen haakjes). Dit gebruik je bv.~bij een citaat, of om in het bijschrift van een overgenomen afbeelding, broncode, tabel, enz. te verwijzen naar de bron. In de volgende paragraaf een voorbeeld van elk.

% \textcite{Knuth1998} schreef een van de standaardwerken over sorteer- en zoekalgoritmen. Experten zijn het erover eens dat cloud computing een interessante opportuniteit vormen, zowel voor gebruikers als voor dienstverleners op vlak van informatietechnologie~\autocite{Creeger2009}.

% Let er ook op: het \texttt{cite}-commando voor de punt, dus binnen de zin. Je verwijst meteen naar een bron in de eerste zin die erop gebaseerd is, dus niet pas op het einde van een paragraaf.
