\chapter{\IfLanguageName{dutch}{Stand van zaken}{State of the art}}%
\label{ch:stand-van-zaken}

% Tip: Begin elk hoofdstuk met een paragraaf inleiding die beschrijft hoe
% dit hoofdstuk past binnen het geheel van de bachelorproef. Geef in het
% bijzonder aan wat de link is met het vorige en volgende hoofdstuk.

% Pas na deze inleidende paragraaf komt de eerste sectiehoofding.

\section{Inleiding}

Dit onderzoek start met een grondige onderdompeling in de wereld van telecommunicatie en technologie om zo een volledige literatuurstudie op te stellen. Verscheidene onderwerpen zullen aan bod komen zoals de geschiedenis van cellulaire netwerken. Hierop volgt een diepgaande bespreking omtrent een LTE netwerk vanwege het feit dat dit het type is dat men zal inzetten gedurende de test opstelling. Een derde hoofdstuk dat volgt is een stuk omtrent verscheidene termen die vaak voorkomen wanneer men de kwaliteit van een bepaald signaal bespreekt. Ten slotte volgt er nog een bespreking van de commerciële applicatie waartegen de proof of concept zal geijkt worden.

\pagebreak

\section{Geschiedenis van het cellulair netwerk}

Gedurende dit hoofdstuk zal de evolutie van verscheidene technologie omtrent cellulaire netwerken besproken worden. Op onderstaande afbeelding \ref{fig:cellularhistory} is een overzicht te zien van de verschillende periodes die besproken zullen worden.

\begin{figure}[!htb]
    \includegraphics[width=1\linewidth]{graphics/cellular_history_timeline}
    \caption[Overzicht omtrent de geschiedenis van cellulaire netwerken]{Geschiedenis omtrent de evolutie van cellulaire netwerken \autocite{Keenan2020}}
    \label{fig:cellularhistory}
\end{figure}

\subsection{Eerste generatie}

De eerste vorm van telecommunicatie kwam er in 1979 onder de naam van 1G. Deze technologie, eerst gebruikt door de inwoners van Japan, bracht de mogelijkheid om anderen te bellen maar het had ook een groot deel aan nadelen. Zo was een gesprek niet beveiligd, was roaming geen optie en de geluidskwaliteit was laag.
De reden dat een gesprek niet beveiligd was vanwege het feit dat er geen encryptie bestond over het netwerk dus eender wie die wou, kon meeluisteren. Deze technologie had een trage downloadsnelheid die specifiek neerkwam op een 2.4 Kbps gemiddeld. \autocite{Galazzo2020}. Om deze snelheid gemakkelijker te visualiseren is het best om een fictieve bestandsoverdracht en de tijd die dit zou nemen te simuleren. Neem nu bijvoorbeeld een korte film van 10 minuten in een MP4 formaat met codec H264 en een resolutie van 1920 x 1080p met een bitrate van 10.000 kbit/s. Zo'n bestand zou 775 MB innemen. \autocite{Helme2019} Gebruik makend van de alle voorgaande informatie over dit bestand en de snelheid van netwerk is de tijd die het zou nemen om het te downloaden neerkomen op 28 dagen, 22 uur, 26 minuten en 40 seconden.

\subsection{Tweede generatie}

2G maakte zijn introductie in de wereld gedurende het jaar 1991 in Finland op de rug van het Global System voor Mobile Communications of afgekort GSM. \autocite{Galazzo2020}. Deze volgende stap definieerde een statistisch grote sprong naar verbetering ten opzichte van de eerste generatie. Men kon namelijk eindelijk berichten versturen naast het al reeds mogelijke bellen. Ook kon men multimedia bestanden delen zoals foto's en waren gesprekken vanaf dit punt wel geëncrypteerd en dus veilig van andermans oren. Doordat de populariteit van connectiviteit ook steeg was er ook een nood aan een efficiëntere toepassing van het spectrum zodat meer apparaten het konden gebruiken. 2G, in tegenstelling tot 1G die niet meer in wereldwijde benutting is, stelt men nog steeds in voor bepaalde toepassingen zoals het Internet of Things. \autocite{Henke2021} Indien men het voorbeeld van de kortfilm uit het eerste hoofdstuk wil herhalen nemen komt de snelheid van een gemiddeld 2G netwerk neet op een 0.2 Mbps. \autocite{Galazzo2020} Hierdoor zou de tijdsduur voor het downloaden van de kortfilm met identieke specificaties als in het eerste voorbeeld gelijkaardig zijn aan 8 uren en 20 seconden. \autocite{Wooding2024}

\subsection{Derde generatie}

De initiële installatie van 3G was voor het eerst functioneel in Japan op de eerste oktober in 2001. De oorsprong hiervan ontstond uit de wederom stijgende vraag van de wereldbevolking om verbonden te zijn met elkaar en verscheidene acties te kunnen uitvoeren terwijl iedereen onderweg was. Dit hield functionaliteiten in zoals videobellen, mails en video's verzenden aan een snelheid vergelijkbaar met snelheden van een vaste computer. Om dit alles mogelijk te maken was er een nood aan hogere data snelheden en ook een beter gebruik van het spectrum om zo meer apparaten toe te laten op het netwerk. \autocite{Dulcey2020} Een technologie die hielp om dit allemaal mogelijk te maken was Code Division Multiple Access of CDMA. Dit functioneert volgens een gespreid spectrum principe. Dit zorgt ervoor dat de elektromagnetische energie tot verdeling kom op een wijdere bandbreedte dan nodig is. Een voordeel van deze technologie is dat het zorgt voor een sterkere beveiliging en het helpt dat de boodschap zo goed als zeker aankomt. Een nadeel van deze technologie is dat data en geluid geen optie hebben om zich tegelijk te verdelen over het netwerk. Dit gevolg zorgt er dus effectief voor dat als iemand aan het bellen is via CDMA er geen optie is om data binnen te halen zoals bijvoorbeeld mails. \autocite{Fendelman2021} Ten slotte om een eenvoudige vergelijking te maken met de andere cellulaire technologieën kan er een gemiddelde data transfer snelheid aannemen van 2 Mbps.\autocite{Galazzo2020} Om dit dan te vertalen naar de kortfilm bij voorgaande voorbeelden zou dit 50 minuten duren om deze data op te halen. \autocite{Wooding2024}

\subsection{Vierde generatie}

Door een nood aan hogere data snelheden werden in 2008 nieuwe standaarden gedefinieerd voor een verbeterde cellulaire technologie, namelijk 4G. Men stelde als één van de doelen dat deze 1000 Mbit/s moest halen op vlak van data transfer snelheden. Een probleem waar men toen op stuitte was dat deze ambities te hoog waren gesteld. Hierdoor volgde een genoodzaakte beslissing in het tot leven roepen van LTE oftewel Long Term Evolution. Dit werd gezien als een tussenstap om zo een betere dienst dan 3G aan te bieden tot men 4G volwaardig kon aanbieden. \autocite{Nicholls2022} Een techniek die toegepast werd in deze iteratie is Orthogonal Frequency Division Multiple Access (OFDMA). Dit werkt door subcarriers, de kleinste eenheid die data kan dragen,  te groeperen in subkanalen. Men verdeelt deze groeperingen van eenheden dan in grotere groepen genaamd bursts die kunnen toegewezen worden aan bepaalde gebruikers. Deze manier van onderverdelen en toewijzen zorgt ervoor dat er een specifieke Quality of Service (QoS) en bandbreedte kan toegewezen worden gebaseerd op wie er verbonden is. Hierdoor word de beschikbare technologie beter ingezet dan bij vorige iteraties. \autocite{Friedmann2007} Indien men de gemiddelde 4G snelheid in Canada in 2020 neemt zoals vermeld door \textcite{Galazzo2020} namelijk 55 Mbps dan zou de eerder gebruikte theoretische data hoeveelheid 1 minuut en 50 seconden nodig hebben om over het netwerk verplaats te kunnen worden. \autocite{Wooding2024}

\subsection{Vijfde generatie}

De laatste vernieuwing op vlak van cellulaire connectiviteit kwam er in 2019 onder de naam van 5G. De noodzaak om nog een sprong te maken inzake data capaciteit, snelheid \& latentie is vanwege de opkomst van nieuwe technologieën zoals VR/AR, autonome auto's en slimme fabrieken. Deze nieuwe partijen vragen om een geavanceerder netwerk dan voorgaand mogelijk was. Ook word er gevraagd voor een betrouwbaarder netwerk, dit zou opgelost kunnen worden door het gebruik van network slicing. Deze techniek bestaat er uit om een fysiek netwerk onder te verdelen in meerdere virtuele netwerken die dan elk een bepaalde doelgroep kunnen toegewezen krijgen. Zo kan één netwerk zorgen voor lage latentie en hoge data betrouwbaarheid. Zo'n opstelling kan van belang zijn voor zelf rijdende auto's waar één seconde informatie bufferen een hoge impact kan hebben. Daar tegenover kan men een virtueel netwerk stellen dat zich eerder toespitst op hoge data volumes maar minder op beschikbaarheid van het netwerk. Dit is geen optie voor de vorige use case maar is wel een optie om bijvoorbeeld films te downloaden. \autocite{Flinders2024} Om een finale vergelijking te maken ten opzichte van de andere technologieën kan de snelheid van dit netwerk weer vertaald worden naar het voorbeeld dat in de voorgaande hoofdstukken gebruikt werd. Hiervoor nemen we een gemiddeld gemeten snelheid in Canada van 169.46 Mbps. \autocite{Galazzo2020} Wanneer dit uitgerekend word komt dit neer op een download tijd van amper 36 seconden. \autocite{Wooding2024}

\section{Architectuur LTE netwerk}
Er kunnen 3 grote delen onderscheiden worden als er vanop een hoog niveau word neergekeken op de architectuur van LTE, namelijk de User Equipment oftewel UE, de Evolved UMTS Terrestrial Radio Access Network mogelijks afgekort tot E-UTRAN en ten slotte de Evolved Packet Core waarnaar verder gerefeerd zal worden als EPC. Deze drie vormen elk een stuk hardware waar dan mogelijks verschillende software op kan geïnstalleerd worden. \autocite{Richard2021} Hieronder volgt een diagram die de visualisatie van het besproken netwerk afbeeld. 
\begin{figure}[!htb]
    \includegraphics[width=1\linewidth]{graphics/LTE_architecture}
    \caption[Overzicht van de verscheidene componenten van de LTE architectuur.]{Diagram van verscheidene componenten en hun samenwerking binnen de LTE architectuur. \autocite{Yunman2021}}
    \label{fig:ltearchitecture}
\end{figure}

Wat een UE is komt simpelweg neer op een apparaat dat door een gebruiker kan gebruikt worden om zich toegang te verschaffen tot een bepaald netwerk. Dit kan gaan van gsm's tot laptops of zelfs slimme horloges of sensoren. \\

Om een uitleg te vormen omtrent de E-UTRAN moet er eerst begrepen worden wat een basis Radio Access Network inhoud vanwege het feit dat een E-UTRAN het vervolg op deze technologie is. Dit onderdeel kan gezien worden als het radio gedeelte om ervoor te zorgen dat eindgebruikers kunnen communiceren met de rest van het systeem. Deze kan men in de omgeving spotten als palen ook wel nodeBs genoemd. Dit kan dan terug onderverdeeld worden in drie componenten namelijk antennes, radio's en baseband units. Een antenne houd zich bezig met het  omzetten van elektrische pulsen naar radio signalen. Radio's daarentegen werken omgekeerd in functionaliteit en zetten van radio signalen naar elektrische pulsen zodat deze in gebruik kunnen genomen worden door andere componenten. Ten slotte zijn er dan nog de baseband units. Het doel van dit onderdeel is om functies aan te bieden zodat de andere componenten logisch kunnen redeneren. \autocite{Jones2021} \\ \\
Om dan terug te komen naar wat de E-UTRAN dan specifiek inhoud kan er een vergelijking gemaakt worden met hoe het er in de voorgaande technologieën aan toe ging. Bij deze was er namelijk een nodeB zoals bij voorgaande besproken maar om deze te kunnen laten werken is er hierbij ook een controller aanwezig. De voornaamste taken van dit object waren het beheren van beschikbare middelen en het bijhouden van de verplaatsingen van een bepaalde eindgebruiker. Dit om zo de verbinding van deze door te kunnen geven van de ene nodeB naar de andere. Bij een  geëvolueerde versie daarvan valt de controller weg en smelt deze samen met de node zorgende voor een eNode. \autocite{Ghayas2019} Het feit dat alle functionaliteit verwoven zit in de node zelf zorgt voor lagere latency doordat er een hechte samenwerking is tussen de verschillende protocol lagen. Ook elimineert het een single point of failure en verlaagt het de kost. \autocite{Palat2011} \\

Nadat een UE verbonden is met een eNodeB word de data doorgestuurd naar de EPC. Dit systeem is een framework die verschillende functionaliteiten ondersteunt. Het verschil met voorgaande technologieën is dat het IP services zijn in plaats van circuit of packet switching. \autocite{Awati2024} Maar alvorens er kan gekeken worden naar wat dit verschil inhoud moet men eerst verstaan waar de twee voorgaande technologieën voor staan. Om te beginnen met circuit switching, dit vanwege het feit dat deze de originele manier was van connecteren bij de eerste telefonie. Deze technologie bestaat eruit om eerst een verbinding op te stellen tussen bron en doel om dan hier data op te verzenden. Deze gebruikers krijgen dan elk een bepaald toegewezen deel van de bandbreedte, wat er voor zorgt dat wanneer men niets verzend de ruimte wel word vrijgehouden wat zorgt voor ferme verspilling. Door de groei in aantal eindgebruikers die het netwerk gebruikten werd de noodzaak groter om over te schakelen van een circuit naar de volgende iteratie namelijk een packet switched netwerk. Hierbij is het de bedoeling om data te verpakken in pakketten bestaande uit een header en de effectieve data. Dit eerste deel zorgt ervoor dat wanneer de data toekomt bij de ontvanger deze ze mogelijk nog kan herorganiseren wanneer ze in een andere volgorde zouden toekomen. Hierdoor hebben ze een netwerk enkel nodig gedurende de tijd dat het pakket onderweg is, wat zorgt dat het veel eerder vrij is voor anderen dan bij de voorgaande technologie. \autocite{BasuMallick2022} Bij IP services is het dan de bedoeling om stem en data te versturen op basis van IP-pakketten die verzonden worden van de ene gebruiker met een toegewezen ip adres en de andere met ook zo'n adres dat ze toegewezen kregen op het moment dat ze voor het eerst verbonden met de cel. Deze verandering zorgt ervoor dat het de netwerken versimpelt terwijl het netwerk hoge performantie en capaciteit behoud. \autocite{Awati2024} \\

Een eerste onderdeel van het EPC is de Mobility Management Entity oftewel MME. Deze procedure is verantwoordelijk voor de authenticatie, autorisatie en de locatie van een eindgebruiker bij te houden. Dit stuk is ook rechtstreeks verbonden met de eNodeBs. Deze groep van base stations is dan weer onderverdeeld in kleinere groepen van Tracking Areas. Deze TA's kan men dan oplijsten in groeperingen van TA's die dicht bij elkaar liggen. Deze TAL is belangrijk vanwege het feit dat de MME bijhoud welke TAL's er waar mogelijk zijn en dat een UE ook uitzend naar buiten met welke TAL ze huidig verbonden zijn. Wanneer de UE dan buiten zijn TAL loopt dan start het een procedure met de MME om een nieuwe TAL toegewezen word. Naast deze info houd de MME ook de laatst verbonden cel bij, dit zodat wanneer men bijvoorbeeld iemand probeert te bellen de MME eerst de laatst verbonden cel contacteert om te zien of de UE kan bereikt worden, indien dit niet lukt vergroten ze hun zoektocht naar de volledige TAL om toch de UE te kunnen bereiken. \autocite{Liou2013} \\

De MME is een onderdeel van het controle gedeelte van de architectuur maar er is ook een niveau dat omgaat met de data te behandelen die de gebruiker wil ophalen of versturen. Een van de componenten die daarvoor instaat is de SGW oftewel Serving Gateway. Dit onderdeel staat in contact met de MME en zorgt voor het doorgeven van data maar ook het verzorgen van de Quality of Service doorheen het data vlak. Men verbind een MME met de SGW omdat wanneer een gebruiker zich voor het eerst registreert bij een netwerk de registratie voltrokken word bij de MME maar deze hierna een signaal moet geven aan de andere componenten om de data stroom ook wel degelijk te starten. Vandaar dat er een link is tussen deze twee elementen. \autocite{Malla2014} \\

Het controle gedeelte van een LTE architectuur bestaat wel niet enkel uit een MME maar bevat ook een Home Subscriber Server (HSS). Deze service bestaat er uit om te fungeren als een databank waarin alle relevante data zich herbergt omtrent de gebruikers. Deze data kan info bevatten dat te maken heeft met locatie, quality of service niveau en roaming rechten indien de gebruiker geen subscriptie heeft bij het huidig netwerk waar het op is aangesloten. Deze component staat rechtstreeks in contact met de MME zodat deze de huidige data kan uitlezen of aanpassen om zo concrete acties uit te kunnen voeren gedurende dat een eindgebruiker verbonden is met het netwerk. \autocite{Shinde2020} \\

Het laatste onderdeel van het data niveau dat tussen de eindgebruiker en het data netwerk zelf staat is de Packet Data Network Gateway (PDN GW). Dit onderdeel krijgt alle data door van de SGW. Dit punt zorgt in principe voor een mogelijkheid om zowel 3GPP, zoals 5G, 4G etc., en niet-3GPP technologieën, zoals WiMax en CDMA 1x of EvDO,  te gebruiken wanneer men wil verbinden met anderen. Een andere functionaliteit waar deze service voor instaat is het filteren van bepaalde pakketten op een gebruikers niveau. Dit doet men door een diepere inspectie te doen op de pakketten die de revue passeren. Doordat men zo'n diepe inspectie doet op wat en wanneer er passeert kan men ook vanuit het gerecht delen onderscheppen en meeluisteren. \autocite{Gilbert2012}      

\section{Data terminologie}

Vooraleer men bepaalde maatstaven bespreekt omtrent de kwaliteit van een specifiek signaal tussen zendmast en eindgebruiker is het belangrijk om een uitleg te geven bij de verschillende meeteenheden die vaak gebruikt worden omtrent deze situaties. Eerst en vooral is er een decibel of dB. In de omgeving van een radio signalen staat deze meeteenheid voor het ratio van verlies of winst dat een bepaald apparaat genereert op een specifiek signaal. Zo kan een kabel voor x dB verlies zorgen van het signaal. Belangrijk bij deze meeteenheid is ook dat de schaal hiervan niet lineair is maar logaritmisch. Wat betekend dat indien een bepaald apparaat de kracht van een signaal halveert, de dB waarde die hieraan gelinkt gaat stijgt of daalt met een stap van 3. Dus neem nu bijvoorbeeld een signaal die door een kabel gaat waar er 9dB verlies op staat, dan betekend dit dat het 3 maal gehalveerd word in kracht alvorens het de kabel verlaat. \autocite{Young2004} \\

Naast een dB waarde dat vooral een ratio is voor hoeveel winst/verlies er zit op een bepaald element is het belangrijk om deze waarde ook te staven ten opzichte van andere standaarden om hieruit verdere bruikbare waarden uit op te halen. Zo kan men het aantal decibel per milliwatt staven onder de naam dBm. Wat dat dus betekent is dat deze maateenheid meet met hoeveel kracht een bepaald signaal word uitgezonden. Hoe lager deze waarde, hoe slechter je signaal. Wel moet men opletten dat wanneer deze waarde te hoog is er een mogelijkheid is dat de kracht van dit signaal er voor zorgt dat ruis en interferentie ook groter worden dat wederom ook zorgt voor een slechter signaal. \autocite{Hardesty2023} \\

Een waarde die vaak voorkomt wanneer men de kwaliteit en sterkte van een netwerk meet is RSSI oftewel Received Signal Strength Indicator. Van al de komende schalen en meetstaven is deze de basis. De numerieke waarde die men hieruit terugkrijgt is namelijk een bepaald punt op een schaal gaande van 0, perfecte verbinding, tot -100, wat staat voor helemaal geen bereik. Wanneer dit gemeten word neemt het veel factoren op in de berekening maar het is niet perfect vanwege het feit dat het geen rekening houd met ruis, onderbrekingen van andere bronnen of hoe betrouwbaar de verbinding wel is.  \autocite{Ramirez2023} \\

Een andere waarde die kan opgemeten wanneer men verbonden is met een bepaalde cel is de RSRP of Reference Signal Received Power. Dit definieert namelijk hoe goed en standvast de verbinding tussen een eindgebruiker en de huidige mast is. De schaal die deze parameter gebruikt is gelijkaardig aan die van RSSI, namelijk dat het negatieve waarden zijn en lager wijst op een slechtere verbinding maar de schaal loopt van -50 tot -130 dBm. De factor die voor een grote impact hierop zorgt is de afstand tot aan de effectieve mast en of de eindgebruiker een duidelijke zichtlijn zonder obstakels heeft met hiervoor vermelde toren. \autocite{Ramirez2023} \\

Gebruik makende van de vorige twee waarden kan men een andere waarde berekenen die meer inzicht geeft in hoe efficiënt het netwerk data oplevert en of er een betrouwbare connectie is. \autocite{Apostu2024} Deze waarde word ook wel RSRQ oftewel Reference Signal Received Quality genoemd en kan berekend worden met een van de volgende formules:

$ RSRP (Watt) = N \times RSRP / RSSQ $ \\
$ RSRP (dBm) = log_{10} (N) + RSRP - RSSI $

In beide van deze is er een nood voor een waarde N waarmee hier gerefereerd wordt naar het nummer van resource blocks aanwezig in de gemeten bandbreedte van de cel waarmee men verbonden is. \autocite{Kovadloff2021} Deze componenten zijn groeperingen van 12 sub carriers wat uiteindelijk uitkomt op een blok van 180 kHz. \autocite{Mishra2018} Op onderstaande figuur kan men zien hoe het aantal beschikbare blokken zich verhoudt tot de verschillende bandbreedtes die momenteel aanwezig zijn op de markt. 

\begin{figure*}[ht]
    \includegraphics[width=1\linewidth]{graphics/bandwithandRB}
    \caption[Relatie tussen bandbreedte en resource blocks]{Relatie tussen bandbreedte en resource blocks \autocite{Kovadloff2021}}
    \label{fig:bandwithandrb}
\end{figure*}

Wanneer men wil bepalen in welke capaciteit een bepaald bericht kan getransporteerd worden over een netwerk is het belangrijk om de ruis aanwezig ook in te calculeren. Een maatstaaf die daarvoor kan ingezet worden is Signal to Noise Ratio (SNR). Dit geeft een ratio weer van de signaalsterkte ten opzichte van de ruis aanwezig. Deze waarde word zoals de voorgaande weergegeven in dBm. Afgaande van hoe het ratio is opgesteld is ook duidelijk dat hoe dichter deze waarde nadert bij de nul, hoe slechter het netwerk er aan toe is en er een lagere kans is dat een bepaald object ook effectief kan verstuurd worden.  \autocite{Sheldon2021}

\section{Commercieel alternatief}

Om de kwaliteit van een proof of concept vast te stellen zal deze vergeleken worden met een alternatief dat al reeds op de markt is en waarvan deze in gebruik is door de doelgroep. Na nader onderzoek blijkt dit G-NetTrack Pro te zijn die in gebruik is door Citymesh. Deze applicatie werd ontwikkeld door Gyokov Solutions om informatie omtrent een netwerk te kunnen verzamelen. Deze data is dan inzetbaar in andere applicaties binnen hun ecosysteem om zo analyses en dergelijke te nemen.\autocite{Solutions2024}  Aanwezige functionaliteiten bevatten maar beperken zich niet tot:

\begin{itemize}
    \item Informatie van een cel verkrijgen, alsook de daar rond liggende cellen
    \item Accurate metingen binnen en buiten.
    \item Ondersteuning voor apparaten die gebruik maken van dual SIM technologie.
    \item Andere manieren om data te representeren zoals een kaart
\end{itemize}

De werking van deze applicatie werkt volgens een simpel principe. De gebruiker start een route en begint met het loggen van de data. Bij het beëindigen van de route stopt men de log en dan slaat het deze informatie op als een .txt bestand in een folder op het apparaat. 

% Dit hoofdstuk bevat je literatuurstudie. De inhoud gaat verder op de inleiding, maar zal het onderwerp van de bachelorproef *diepgaand* uitspitten. De bedoeling is dat de lezer na lezing van dit hoofdstuk helemaal op de hoogte is van de huidige stand van zaken (state-of-the-art) in het onderzoeksdomein. Iemand die niet vertrouwd is met het onderwerp, weet nu voldoende om de rest van het verhaal te kunnen volgen, zonder dat die er nog andere informatie moet over opzoeken \autocite{Pollefliet2011}.

% Je verwijst bij elke bewering die je doet, vakterm die je introduceert, enz.\ naar je bronnen. In \LaTeX{} kan dat met het commando \texttt{$\backslash${textcite\{\}}} of \texttt{$\backslash${autocite\{\}}}. Als argument van het commando geef je de ``sleutel'' van een ``record'' in een bibliografische databank in het Bib\LaTeX{}-formaat (een tekstbestand). Als je expliciet naar de auteur verwijst in de zin (narratieve referentie), gebruik je \texttt{$\backslash${}textcite\{\}}. Soms is de auteursnaam niet expliciet een onderdeel van de zin, dan gebruik je \texttt{$\backslash${}autocite\{\}} (referentie tussen haakjes). Dit gebruik je bv.~bij een citaat, of om in het bijschrift van een overgenomen afbeelding, broncode, tabel, enz. te verwijzen naar de bron. In de volgende paragraaf een voorbeeld van elk.

% \textcite{Knuth1998} schreef een van de standaardwerken over sorteer- en zoekalgoritmen. Experten zijn het erover eens dat cloud computing een interessante opportuniteit vormen, zowel voor gebruikers als voor dienstverleners op vlak van informatietechnologie~\autocite{Creeger2009}.

% Let er ook op: het \texttt{cite}-commando voor de punt, dus binnen de zin. Je verwijst meteen naar een bron in de eerste zin die erop gebaseerd is, dus niet pas op het einde van een paragraaf.
