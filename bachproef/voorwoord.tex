%%=============================================================================
%% Voorwoord
%%=============================================================================

\chapter*{\IfLanguageName{dutch}{Woord vooraf}{Preface}}%
\label{ch:voorwoord}

%% TODO:
%% Het voorwoord is het enige deel van de bachelorproef waar je vanuit je
%% eigen standpunt (``ik-vorm'') mag schrijven. Je kan hier bv. motiveren
%% waarom jij het onderwerp wil bespreken.
%% Vergeet ook niet te bedanken wie je geholpen/gesteund/... heeft

Als tiener zag ik de opkomst van allerlei verschillende applicaties die een gebruiker de toegang gaven tot een volledig nieuwe wereld waarvan de limieten gelijk waren aan de verbeelding van degene die dit in elkaar had gestoken. Hierdoor groeide mijn zin om zelf ook aan projecten te werken en applicaties op de markt te brengen die mensen helpen en een impact op de wereld kunnen hebben. Toen Citymesh mij benaderde met een probleem zag ik een duidelijke kans waarbij ik mijn droom kon omzetten in een realiteit. \\

Voor deze prachtige mogelijkheid wil ik Jens Buysse bedanken voor het vertrouwen en de kans om aan dit idee te werken . Voor de steun en advies tijdens mijn hobbelig pad wil ik alsook graag Sion Verschraege bedanken. Voor de extra hulp en het geloof in mij dat ik een bruikbare applicatie zou afleveren wil ik graag ook het Research team bij Citymesh bedanken. Ten slotte ook een welgemeende dankuwel aan mijn familie voor het ondersteunen van mij gedurende mijn volledige schooltraject. \\

Het feit dat de wereld van cellulaire technologie een nieuw gegeven was voor mij gaf de doorslag omtrent mijn motivatie om aan dit onderzoek te beginnen . Dit traject was een prachtige afsluiter van een 4 jaar durend avontuur waarbij ik zelf een transformatie onderging van jong onwetende student die dacht dat hij omweg kon met een computer tot een competente applicatie ontwikkelaar.