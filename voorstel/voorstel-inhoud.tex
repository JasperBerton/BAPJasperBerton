%---------- Inleiding ---------------------------------------------------------

\section{Introductie}%
\label{sec:introductie}

\subsection{Kaderen thema}

1.212 bedrijven beschikten over minimum 1 privaat mobiel netwerk in 2023. Voorspeld word ook dat dit nummer ferm zal toenemen wanneer landen in de toekomst spectrum beschikbaar stellen specifiek voor bedrijven. \autocite{Dux2023} Bij deze snel groeiende markt steekt zich wel de problematiek op van een nood naar een specifieke toolset om gerichte testen te kunnen uitvoeren.

\subsection{De doelgroep}

De groep waarop zal gefocust worden gedurende deze studie is private bedrijven die zich inzetten op het ontwikkelen van private netwerken voor andere bedrijven, met name Citymesh. Dit bedrijf is al actief sinds 2016 en specialiseert zich in het uitbouwen van netwerken voor andere bedrijven. \autocite{Cassauwers2019}

\subsection{Probleemstelling en onderzoeksvraag}

Een kenmerk van private netwerken is dat ze afgestemd zijn op de industrie waarin ze toegepast zullen worden. \autocite{Alen2020} Een nadeel aan deze gepersonaliseerde manier van aanpak is dat een gestandaardiseerde test voor commerciële netwerken niet gebruikt kan worden voor deze netwerken. De huidige status van deze test opstelling bestond hierdoor uit een grote partitie aan handmatig werk en verscheidene omwegen om de verwonnen data te gebruiken.

\subsection{De onderzoeksdoelstelling}

Het uiteindelijke doel van deze studie is om een proof-of-concept af te leveren. Deze PoC zou een Android applicatie zijn die kan ingezet worden gedurende tijdelijke opstellingen van private netwerken op evenementen. Dit met als doel om de uit te voeren routine te automatiseren en standaardiseren zodat er meer en grondiger data kan verzameld worden. Deze verwonnen info zou men dan kunnen inzetten om het netwerk beter te diagnosen en rapporten voor klanten uit op te maken.

%---------- Stand van zaken ---------------------------------------------------

\section{Literatuurstudie}%
\label{sec:state-of-the-art}

\subsection{LTE netwerk architectuur}

Een Long Term Evolution Netwerk oftewel LTE, is een technologie die is opgebouwd uit verschillende componenten. Deze zijn stuk voor stuk eigen aan deze iteratie van cellulaire systemen. Verder in dit hoofdstuk zal besproken worden hoe de verschillende elementen zorgen dat de End User (UE) zich kan verbinden met de rest van de wereld.\\

Een van de belangrijkste onderdelen van deze opstelling is de core. Verantwoordelijk voor het beslissen over wat er gebeurt met een UE zijn er verschillende logische onderdelen. Dit systeem bestaat er vooral uit om abonnees te registreren en daarna hun data stromen elk een niveau van Quality Of Service toe te kennen. Deze vooraf gedefinieerde QoS niveaus zorgen ervoor dat de nodige middelen correct worden toegekend. Zo zijn de pakketten en de stroom van data volledig anders wanneer een eindgebruiker een video live bekijkt ten opzichte van een telefoongesprek dat via Voice over IP gaat. \autocite{Palat2011} \\

Het ander hoofdcomponent in een LTE-netwerk is het Radio Access Network (RAN). Dit onderdeel van het netwerk verzorgt de connectie tussen de core, en de EU. Bij LTE bestaat dit deel uit een eNodeBs. In bepaalde opstellingen kunnen er meerdere van deze torens verbonden zijn met één core. Wat verschillend is tussen deze elementen en degene van de voorgaande generatie is dat er geen gecentraliseerde controle unit is. Lage latency en verhoogde efficiëntie zijn hier een gevolg van, maar dit zorgt er wel voor dat alle data van de EU moet verplaatsen van eNodeB naar een andere wanneer deze zich verplaatst. \autocite{Palat2011} \\

\subsection{Gamificatie en sport}

Gamificatie is al reeds aanwezig in verschillende sport applicaties. Bij meesten, zoals bijvoorbeeld bij Strava, is er het gebruik van verschillende vriendelijke competities waar een top 3 of top 10 word gevisualiseerd. De keuze voor deze manier van aanpak kan wel negatieve gevolgen hebben omdat het er voor kan zorgen dat mensen gedemotiveerd raken doordat ze buiten die top 10 of zelfs top 20 vallen. \autocite{Alen2020} Het is dus belangrijk om oftewel te zorgen dat de speler ten alle tijden zich betrokken voelt tot het spel element ofdat er een gezamenlijk doel is om naar samen te werken.

\subsection{Strava API}

De reden waarom gekozen is voor Strava is vanwege het feit dat Strava niet platform gebonden is doordat het een applicatie is die niet gebonden is aan externe bepaalde particuliere hardware. \autocite{David2023} Hierdoor is er geen drempel voor nieuwe sporters of uitsluiting op basis van materiaal dat de iemand al dan niet bezit. 

Een hoeksteen waar de applicatie zwaar op zal steunen is de API aangeleverd door de sport app Strava. Strava zelf is een applicatie gebruikt door sportievelingen om informatie van hun sportprestaties bij te houden. \\

De API aangeboden door dit bedrijf is gratis maar als ontwikkelaar heb je wel een account nodig. Data kan niet opgehaald worden in grote getallen van eender wie zijn account maar elke gebruiker moet zich aanmelden met zijn of haar eigen account via 0Auth 2.0. Er zijn ook enkele limitaties aan de API namelijk request limits van 200 per 15 minuten en een maximum van 2000 requests per dag. \autocite{Strava2023}

\section{Methodologie}%
\label{sec:methodologie}

\subsection{Deelnemers}

De deelnemers van deze studie zullen bestaan uit verschillende medewerkers van het Research en Flex team binnen Citynesh die zullen ingezet worden om de applicatie grondig te testen en data te verzamelen.

\subsection{Ontwerp en procedure}

Voor dit onderzoek is het de bedoeling om uiteindelijk tot een proof-of-concept te geraken dat kan afgeleverd worden. Hiervoor werden de volgende stappen onderverdeeld in bepaalde fasen:

\begin{enumerate}
    \item \hyperref[subsub:beschrijving]{Beschrijving casus}
    \item \hyperref[subsub:uitwerking]{Uitwerken POC}
    \item \hyperref[subsub:testfase]{Testen van de applicatie}
    \item \hyperref[subsub:conclusies]{Conclusies}
\end{enumerate}

\subsection{Beschrijving casus}
\label{subsub:beschrijving}

Gedurende deze eerste fase is het belangrijk om informatie te verzamelen omtrent hoe het gamificatie aspect verwerkt zal worden in de applicatie. Ook een achtergrondstudie naar het grondprobleem dat zal worden aangepakt en wat de beste manieren zijn om dit probleem aan te pakken vormen een groot aspect van deze fase. \\

\subsection{Uitwerken Proof-of-Concept}
\label{subsub:uitwerking}

Deze tweede fase zal ook degene worden waar de meeste tijd in zal kruipen. Dit deel bestaat er namelijk uit om de volledige POC uit te werken. Deze applicatie zal als front-end functioneel uitgebouwd worden met React en als back-end zal het rechtstreeks steunen op de Strava API. 

\subsection{Testen van de applicatie}
\label{subsub:testfase}

Gedurende deze fase zal de applicatie aan automatische testen onderworpen worden in verband met beveiliging en veiligheid. Daarna kan de applicatie open gesteld worden voor een selecte groep product testers die de kans krijgen het product uit te testen en daarna via een formulier de kans krijgen om hun mening en ervaringen op een gestructureerde manier door te geven zodat deze dan kunnen verwerkt worden. De data die hier centraal staat is de gebruikstijden van de applicatie over verloop van tijd als ook het aantal afgelegde kilometers per test persoon over het verloop van tijd. Dit zal dan uiteindelijk aangevuld worden met een persoonlijke enquete waar de ervaringen van de testers in zal opgenomen worden. 

\subsection{Conclusies}
\label{subsub:conclusies}

Als laatste fase is het belangrijk om te reflecteren op het gehele proces en de data te verwerken die verzameld is doorheen de fase ervoor. Hieruit kan dan een conclusie opgemaakt worden. Dit kan dan dienen als advies voor anderen naar de toekomst toe. 



%---------- Verwachte resultaten ----------------------------------------------
\section{Verwacht resultaat, conclusie}%
\label{sec:verwachte_resultaten}

Uit de verwachte resultaten kan opgemaakt worden of er een bruikbare mogelijkheid is om een applicatie te ontwikkelen om private netwerken grondig en bruikbaar te kunnen testen. Verwacht word dat er wel degelijk een applicatie kan ontwikkeld worden maar dat er beperkingen zullen zijn naarmate de diepgang en personalisatie in de applicatie en het automatiseren van de routine. Ook is er een kans dat door de resultaten in real-time te versturen naar een online databank, dit een invloed heeft op de gemeten resultaten.
