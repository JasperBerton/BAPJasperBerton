%---------- Inleiding ---------------------------------------------------------

\section{Introductie}%
\label{sec:introductie}

\subsection{Kaderen thema}

Uit cijfers van de Gezondheidsenquete opgemaakt in 2018 blijkt dat 15.9 procent van de volwassen bevolking obees is. ~\autocite{Drieskens2018}.  Dit getal is ook in een steeds stijgende lijn wat een problematisch gegeven is. 

\subsection{De doelgroep}

De groep waarop zal gefocust worden gedurende deze studie is Vlaamse jongeren tussen de leeftijd van 16-21 jaar. Dit vanwege het feit dat jongeren van deze leeftijdsgroep al veel ervaring hebben met het gebruik van hun telefoon gedurende fitness activiteiten. Ook omdat het onderwerp van gamificatie het best kan aanslaan bij jongeren die nog staan te springen voor deze manier van aanpak.

\subsection{Probleemstelling en onderzoeksvraag}

Gamificatie kan gedefinieerd worden als het introduceren van spel elementen in het echte leven. \autocite{Deterding2011} Door deze aspecten over te nemen kan de saaie of repetitieve taak die een bepaalde persoon moet ondernemen in een ander licht gezet worden met bijvoorbeeld trofeëen of levels. Dit zou dan op zijn beurt kunnen zorgen dat iemand een bepaalde taak dus vaker uitvoert en het langer volhoud om dit repetitief te herhalen. De uitdagingen die hierbij gepaard kunnen gaan is de aandacht van de sporter te behouden en interesse te wekken tot het gebruik van de applicatie. 

\subsection{De onderzoeksdoelstelling}

Het uiteindelijke doel van deze studie is om een proof-of-concept af te leveren. Deze PoC zou een Android applicatie zijn die gebruik maakt van de sport app Strava als een backend/API. Dit zou dan uitgebracht kunnen worden naar het bredere publiek om dan zo meer mensen te motiveren om meer te sporten. Naast de proof-of-concept zal er ook een literatuurstudie en een volledige documentatie zijn van het proces.

%---------- Stand van zaken ---------------------------------------------------

\section{Literatuurstudie}%
\label{sec:state-of-the-art}

\subsection{Geschiedenis gamificatie}

Een van de eerste en meest succesvolle projecten waarbij gamificatie werd toegepast was door het bedrijf Sperry en Hutchinson Co. opgericht in 1896 te Alabama. Hun idee om mensen te motiveren met spel elementen hebben ze geïmplementeerd door het invoeren van stempels waarmee klanten dan producten konden kopen. \autocite{Christians2018}. \\

Daarna duurde het een hele periode alvorens er een evolutie kwam in dit gebied. Gamificatie zoals we het vandaag de dag kennen werd geboren in het jaar 2003. Nick Pelling kwam namelijk af met het idee om speelvolle elementen toe te voegen aan bepaalde producten. Zijn idee was sterk en de term was geboren maar zijn bedrijf ging uiteindelijk ten onder. \autocite{Khaitova2021} \\

De volgende evolutie kwam er in 2010 toen Jane McGonigal opperde dat gamers de wereld konden veranderen als ze maar de juiste richting in gedirigeerd werden. Door de positieve ontvangst van het publiek leidde deze uitspraak tot de eerste Gamification Summit een jaar later. \autocite{Christians2018} \\

Hierna veranderde er niet veel meer aan het medium maar kwamen er wel nog veel andere spelers op de markt en geraakte de term wijdverspreid.

\subsection{Strava API}

Een hoeksteen waar de applicatie zwaar op zal steunen is de API aangeleverd door de sport app Strava. Strava zelf is een applicatie gebruikt door sportievelingen om informatie van hun sportprestaties bij te houden. \\

De API aangeboden door dit bedrijf is gratis maar als ontwikkelaar heb je wel een account nodig. Data kan niet opgehaald worden in grote getallen van eender wie zijn account maar elke gebruiker moet zich aanmelden met zijn of haar eigen account via 0Auth 2.0. Er zijn ook enkele limitaties aan de API namelijk request limits van 200 per 15 minuten en een maximum van 2000 requests per dag. \autocite{Strava2023}

\section{Methodologie}%
\label{sec:methodologie}

\subsection{Deelnemers}

De deelnemers van deze studie zullen bestaan uit verschillende jongeren tussen 16-21 jaar als testers voor de applicatie. Dit omdat zij de uiteindelijke eindgebruikers zullen zijn van de applicatie moest deze in circulatie gezet worden.

\subsection{Ontwerp en procedure}

Voor dit onderzoek is het de bedoeling om uiteindelijk tot een proof-of-concept te geraken dat kan afgeleverd worden. Hiervoor werden de volgende stappen onderverdeeld in bepaalde fasen:

\begin{enumerate}
    \item \hyperref[subsub:beschrijving]{Beschrijving casus}
    \item \hyperref[subsub:uitwerking]{Uitwerken POC}
    \item \hyperref[subsub:testfase]{Testen van de applicatie}
    \item \hyperref[subsub:conclusies]{Conclusies}
\end{enumerate}

\subsection{Beschrijving casus}
\label{subsub:beschrijving}

Gedurende deze eerste fase is het belangrijk om informatie te verzamelen omtrent hoe het gamificatie aspect verwerkt zal worden in de applicatie. Ook een achtergrondstudie naar het grondprobleem dat zal worden aangepakt en wat de beste manieren zijn om dit probleem aan te pakken vormen een groot aspect van deze fase. \\

\subsection{Uitwerken Proof-of-Concept}
\label{subsub:uitwerking}

Deze tweede fase zal ook degene worden waar de meeste tijd in zal kruipen. Dit deel bestaat er namelijk uit om de volledige POC uit te werken. Deze applicatie zal als front-end functioneel uitgebouwd worden met React en als back-end zal het rechtstreeks steunen op de Strava API. 

\subsection{Testen van de applicatie}
\label{subsub:testfase}

Gedurende deze fase zal de applicatie aan automatische testen onderworpen worden in verband met beveiliging en veiligheid. Daarna kan de applicatie open gesteld worden voor een selecte groep product testers die de kans krijgen het product uit te testen en daarna via een formulier de kans krijgen om hun mening en ervaringen op een gestructureerde manier door te geven zodat deze dan kunnen verwerkt worden. 

\subsection{Conclusies}
\label{subsub:conclusies}

Als laatste fase is het belangrijk om te reflecteren op het gehele proces en de data te verwerken die verzameld is doorheen de fase ervoor. Hieruit kan dan een conclusie opgemaakt worden. Dit kan dan dienen als advies voor anderen naar de toekomst toe. 



%---------- Verwachte resultaten ----------------------------------------------
\section{Verwacht resultaat, conclusie}%
\label{sec:verwachte_resultaten}

Uit de verwachte resultaten kan opgemaakt worden of een gamificatie van sportprestaties een nuttige manier is om verder te gaan om zo meer mensen te motiveren om regelmatiger te sporten. Verwacht is dat door de stimulatie aan hand van medailles en levels dat meer mensen gemotiveerder zullen zijn door een gevoel te hebben dat ze ergens naartoe werken dat reeël en duidelijk afgebakend is.
