%---------- Inleiding ---------------------------------------------------------

\section{Introductie}%
\label{sec:introductie}

\subsection{Kaderen thema}

Uit cijfers van de Gezondheidsenquete opgemaakt in 2018 blijkt dat 15.9 procent van de volwassen bevolking obees is. ~\autocite{Drieskens2018}.  Dit getal is ook in een steeds stijgende lijn wat een problematisch gegeven is. 

\subsection{De doelgroep}

De groep waarop zal gefocust worden gedurende deze studie is Vlaamse jongeren tussen de leeftijd van 16-21 jaar. Dit vanwege het feit dat jongeren van deze leeftijdsgroep al veel ervaring hebben met het gebruik van hun telefoon gedurende fitness activiteiten. Ook omdat het onderwerp van gamificatie het best kan aanslaan bij jongeren die nog staan te springen voor deze manier van aanpak.

\subsection{Probleemstelling en onderzoeksvraag}

Gamificatie kan gedefinieerd worden als het introduceren van spel elementen in het echte leven. \autocite{Deterding2011} Door deze aspecten over te nemen kan de saaie of repetitieve taak die een bepaalde persoon moet ondernemen in een ander licht gezet worden met bijvoorbeeld trofeëen of levels. Dit zou dan op zijn beurt kunnen zorgen dat iemand een bepaalde taak dus vaker uitvoert en het langer volhoud om dit repetitief te herhalen. De uitdagingen die hierbij gepaard kunnen gaan is de aandacht van de sporter te behouden en interesse te wekken tot het gebruik van de applicatie. 

\subsection{De onderzoeksdoelstelling}

Het uiteindelijke doel van deze studie is om een proof-of-concept af te leveren. Deze PoC zou een Android applicatie zijn die gebruik maakt van de sport app Strava als een backend/API. Dit zou dan uitgebracht kunnen worden naar het bredere publiek om dan zo meer mensen te motiveren om meer te sporten. Naast de proof-of-concept zal er ook een literatuurstudie en een volledige documentatie zijn van het proces.

%---------- Stand van zaken ---------------------------------------------------

\section{Literatuurstudie}%
\label{sec:state-of-the-art}

\subsection{Geschiedenis gamificatie}

Een van de eerste en meest succesvolle projecten waarbij gamificatie werd toegepast was door het bedrijf Sperry en Hutchinson Co. opgericht in 1896 te Alabama. Hun idee om mensen te motiveren met spel elementen hebben ze geïmplementeerd door het invoeren van stempels waarmee klanten dan producten konden kopen. \autocite{Christians2018}. \\

Daarna duurde het een hele periode alvorens er een evolutie kwam in dit gebied. Gamificatie zoals we het vandaag de dag kennen werd geboren in het jaar 2003. Nick Pelling kwam namelijk af met het idee om speelvolle elementen toe te voegen aan bepaalde producten. Zijn idee was sterk en de term was geboren maar zijn bedrijf ging uiteindelijk ten onder. \autocite{Khaitova2021} \\

De volgende evolutie kwam er in 2010 toen Jane McGonigal opperde dat gamers de wereld konden veranderen als ze maar de juiste richting in gedirigeerd werden. Door de positieve ontvangst van het publiek leidde deze uitspraak tot de eerste Gamification Summit een jaar later. \autocite{Christians2018}

Hierna veranderde er niet veel meer aan het medium maar kwamen er wel nog veel andere spelers op de markt en geraakte de term wijdverspreid.




\section{Methodologie}%
\label{sec:methodologie}

Hier beschrijf je hoe je van plan bent het onderzoek te voeren. Welke onderzoekstechniek ga je toepassen om elk van je onderzoeksvragen te beantwoorden? Gebruik je hiervoor literatuurstudie, interviews met belanghebbenden (bv.~voor requirements-analyse), experimenten, simulaties, vergelijkende studie, risico-analyse, PoC, \ldots?

Valt je onderwerp onder één van de typische soorten bachelorproeven die besproken zijn in de lessen Research Methods (bv.\ vergelijkende studie of risico-analyse)? Zorg er dan ook voor dat we duidelijk de verschillende stappen terug vinden die we verwachten in dit soort onderzoek!

Vermijd onderzoekstechnieken die geen objectieve, meetbare resultaten kunnen opleveren. Enquêtes, bijvoorbeeld, zijn voor een bachelorproef informatica meestal \textbf{niet geschikt}. De antwoorden zijn eerder meningen dan feiten en in de praktijk blijkt het ook bijzonder moeilijk om voldoende respondenten te vinden. Studenten die een enquête willen voeren, hebben meestal ook geen goede definitie van de populatie, waardoor ook niet kan aangetoond worden dat eventuele resultaten representatief zijn.

Uit dit onderdeel moet duidelijk naar voor komen dat je bachelorproef ook technisch voldoen\-de diepgang zal bevatten. Het zou niet kloppen als een bachelorproef informatica ook door bv.\ een student marketing zou kunnen uitgevoerd worden.

Je beschrijft ook al welke tools (hardware, software, diensten, \ldots) je denkt hiervoor te gebruiken of te ontwikkelen.

Probeer ook een tijdschatting te maken. Hoe lang zal je met elke fase van je onderzoek bezig zijn en wat zijn de concrete \emph{deliverables} in elke fase?

%---------- Verwachte resultaten ----------------------------------------------
\section{Verwacht resultaat, conclusie}%
\label{sec:verwachte_resultaten}

Hier beschrijf je welke resultaten je verwacht. Als je metingen en simulaties uitvoert, kan je hier al mock-ups maken van de grafieken samen met de verwachte conclusies. Benoem zeker al je assen en de onderdelen van de grafiek die je gaat gebruiken. Dit zorgt ervoor dat je concreet weet welk soort data je moet verzamelen en hoe je die moet meten.

Wat heeft de doelgroep van je onderzoek aan het resultaat? Op welke manier zorgt jouw bachelorproef voor een meerwaarde?

Hier beschrijf je wat je verwacht uit je onderzoek, met de motivatie waarom. Het is \textbf{niet} erg indien uit je onderzoek andere resultaten en conclusies vloeien dan dat je hier beschrijft: het is dan juist interessant om te onderzoeken waarom jouw hypothesen niet overeenkomen met de resultaten.

