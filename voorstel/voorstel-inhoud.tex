%---------- Inleiding ---------------------------------------------------------

\section{Introductie}%
\label{sec:introductie}

\subsection{Kaderen thema}

1.212 bedrijven beschikten over minimum 1 privaat mobiel netwerk in 2023. Voorspeld word ook dat dit nummer ferm zal toenemen wanneer landen in de toekomst spectrum beschikbaar stellen specifiek voor bedrijven. \autocite{Dux2023} Bij deze snel groeiende markt steekt zich wel de problematiek op van een nood naar een specifieke toolset om gerichte testen te kunnen uitvoeren.

\subsection{De doelgroep}

De groep waarop zal gefocust worden gedurende deze studie is private bedrijven die zich inzetten op het ontwikkelen van connectiviteit voor andere bedrijven, met name Citymesh. Dit bedrijf is al actief sinds 2016 en specialiseert zich in het uitbouwen van netwerken voor andere bedrijven. \autocite{Cassauwers2019}

\subsection{Probleemstelling en onderzoeksvraag}

Een kenmerk van private netwerken is dat ze afgestemd zijn op de industrie waarin ze toegepast zullen worden. \autocite{Alen2020} Een nadeel aan deze gepersonaliseerde manier van aanpak is dat een gestandaardiseerde test voor commerciële netwerken niet gebruikt kan worden voor deze netwerken. De huidige status van deze test opstelling bestond hierdoor uit een grote partitie aan handmatig werk en verscheidene omwegen om de verwonnen data te gebruiken.

\subsection{De onderzoeksdoelstelling}

Het uiteindelijke doel van deze studie is om een proof-of-concept af te leveren. Deze PoC zou een Android applicatie zijn die kan ingezet worden gedurende tijdelijke opstellingen van private netwerken op evenementen. Dit met als doel om de uit te voeren routine te automatiseren en standaardiseren zodat er meer en grondiger data kan verzameld worden. Deze verwonnen info zou men dan kunnen inzetten om het netwerk beter te diagnosen en rapporten voor klanten uit op te maken.

%---------- Stand van zaken ---------------------------------------------------

\section{Literatuurstudie}%
\label{sec:state-of-the-art}

\subsection{LTE netwerk architectuur}

Een Long Term Evolution Netwerk oftewel LTE, is een technologie die is opgebouwd uit verschillende componenten. Deze zijn stuk voor stuk eigen aan deze iteratie van cellulaire systemen. Verder in dit hoofdstuk zal besproken worden hoe de verschillende elementen zorgen dat de End User (UE) zich kan verbinden met de rest van de wereld.\\

Een van de belangrijkste onderdelen van deze opstelling is de core. Verantwoordelijk voor het beslissen over wat er gebeurt met een UE zijn er verschillende logische onderdelen. Dit systeem bestaat er vooral uit om abonnees te registreren en daarna hun data stromen elk een niveau van Quality Of Service toe te kennen. Deze vooraf gedefinieerde QoS niveaus zorgen ervoor dat de nodige middelen correct worden toegekend. Zo zijn de pakketten en de stroom van data volledig anders wanneer een eindgebruiker een video live bekijkt ten opzichte van een telefoongesprek dat via Voice over IP gaat. \autocite{Palat2011} \\

Het ander hoofdcomponent in een LTE-netwerk is het Radio Access Network (RAN). Dit onderdeel van het netwerk verzorgt de connectie tussen de core, en de EU. Bij LTE bestaat dit deel uit een eNodeBs. In bepaalde opstellingen kunnen er meerdere van deze torens verbonden zijn met één core. Wat verschillend is tussen deze elementen en degene van de voorgaande generatie is dat er geen gecentraliseerde controle unit is. Lage latency en verhoogde efficiëntie zijn hier een gevolg van, maar dit zorgt er wel voor dat alle data van de EU moet verplaatsen van eNodeB naar een andere wanneer deze zich verplaatst. \autocite{Palat2011} \\

\subsection{Publiek versus privaat}

TODO.

\subsection{Telecom spectrum licensing}

Om het opzetten van private netwerken mogelijk te maken kunnen mobiele netwerk operatoren, oftewel MNO's, een licensie aankopen zodat ze een bepaald deel van het elektromagnetisch spectrum kunnen gebruiken voor eigen gebruik. Het is belangrijk dat dit sterk gerugeleerd is om er zo zeker van te zijn dat er geen interferentie is tussen verscheidene uitgezonden signalen binnen een bepaald geografisch gebied.\autocite{Trick2022} \\

Unlicensed oftewel gedeeld spectrum daarentegen kan gebruikt worden door alle operatoren maar hierdoor komt het wel voor dat gebruikers een slechtere ervaring hebben doordat het spectrum gedeeld moet worden onder alle gebruikers ten opzichte van een gelimiteerd aantal dat toegelaten word binnen een gelicensieërd spectrum.\autocite{Trick2022}


\section{Methodologie}%
\label{sec:methodologie}

\subsection{Deelnemers}

De deelnemers van deze studie zullen bestaan uit verschillende medewerkers van het Research en Flex team binnen Citynesh die zullen ingezet worden om de applicatie grondig te testen en data te verzamelen.

\subsection{Ontwerp en procedure}

Voor dit onderzoek is het de bedoeling om uiteindelijk tot een proof-of-concept te geraken dat kan afgeleverd worden. Hiervoor werden de volgende stappen onderverdeeld in bepaalde fasen:

\begin{enumerate}
    \item \hyperref[subsub:beschrijving]{Beschrijving casus}
    \item \hyperref[subsub:uitwerking]{Uitwerken POC}
    \item \hyperref[subsub:testfase]{Testen van de applicatie}
    \item \hyperref[subsub:conclusies]{Conclusies}
\end{enumerate}

\subsection{Beschrijving casus}
\label{subsub:beschrijving}

Gedurende deze eerste fase is het belangrijk om informatie te verzamelen omtrent hoe een privaat netwerk functioneert om zo routines uit te schrijven zodat het kan verwerkt worden in de applicatie. Ook een achtergrondstudie naar de huidige state of the art en de industrie dat zal worden aangepakt en wat de beste manieren zijn om dit probleem aan te pakken vormen een groot aspect van deze fase. \\

\subsection{Uitwerken Proof-of-Concept}
\label{subsub:uitwerking}

Deze fase loopt gelijkdurig met de andere fases doordat het een langdurig proces is. Dit bestaat er namelijk uit om de een Proof-Of-Concept uit te werken. Deze applicatie word intern uitgebouwd binnen het Research team als een Android applicatie steunend op een open-source tijdreeksdatabase genaamd Influx waarvoor deze data dan kan gebruikt worden in een Grafana dashboard.

\subsection{Testen van de applicatie}
\label{subsub:testfase}

Doorheen verscheidene sprint iteraties van de applicatie zal het onderworpen worden aan testen op zowel verscheidene publieke als private netwerken. Dit zorgt er eenderhand voor dat er uitgebreider kan getest worden op de stabiliteit en het functioneren van de verscheidene netwerken. Een bijproduct van deze fase is dat er data verworven wordt waarbij de verscheidene aangeboden diensten kunnen vergeleken worden voor een gegeven locatie. De code integriteit zal ook onderworpen worden aan verscheidene testen door het implementeren van een cicd pipeline.

\subsection{Conclusies}
\label{subsub:conclusies}

Als laatste fase is het belangrijk om te reflecteren op het gehele proces en de data te verwerken die verzameld is doorheen de fase ervoor. Hieruit kan dan een presentatie opgemaakt worden omtrent de mogelijke voordelen die de proof-of-concept brengt op het testen van een netwerk. Belangrijk is ook om alle betrokken partijen te betrekken bij het beoordelen van de applicatie die door hen is gebruikt. 



%---------- Verwachte resultaten ----------------------------------------------
\section{Verwacht resultaat, conclusie}%
\label{sec:verwachte_resultaten}

Uit de verwachte resultaten kan opgemaakt worden of er een bruikbare mogelijkheid is om een applicatie te ontwikkelen om private netwerken grondig en bruikbaar te kunnen testen. Verwacht word dat er wel degelijk een applicatie kan ontwikkeld worden maar dat er beperkingen zullen zijn naarmate de diepgang en personalisatie in de applicatie en het automatiseren van de routine. Ook is er een kans dat door de resultaten in real-time te versturen naar een online databank, dit een invloed heeft op de gemeten resultaten. Een van de belangrijkste taken die de proof-of-concept zou moeten volbrengen is het reduceren van de nodige tijd om een netwerk te testen met bruikbare resultaten. 
