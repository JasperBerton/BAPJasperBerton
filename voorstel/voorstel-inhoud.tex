%---------- Inleiding ---------------------------------------------------------

\section{Introductie}%
\label{sec:introductie}

\subsection{Kaderen thema}

1.212 bedrijven beschikten over minimum 1 privaat mobiel netwerk in 2023. Voorspeld word ook dat dit nummer ferm zal toenemen wanneer landen in de toekomst spectrum beschikbaar stellen specifiek voor bedrijven. \autocite{Dux2023} Bij deze snel groeiende markt steekt zich wel de problematiek op van een nood naar een specifieke toolset om gerichte testen te kunnen uitvoeren.

\subsection{De doelgroep}

De groep waarop zal gefocust worden gedurende deze studie is private bedrijven die zich inzetten op het ontwikkelen van private netwerken voor andere bedrijven, met name Citymesh. Dit bedrijf is al actief sinds 2016 en specialiseert zich in het uitbouwen van netwerken voor andere bedrijven. \autocite{Cassauwers2019}

\subsection{Probleemstelling en onderzoeksvraag}



\subsection{De onderzoeksdoelstelling}

Het uiteindelijke doel van deze studie is om een proof-of-concept af te leveren. Deze PoC zou een Android applicatie zijn die gebruik maakt van de sport app Strava als een backend/API. Dit zou dan uitgebracht kunnen worden naar het bredere publiek om dan zo meer mensen te motiveren om meer te sporten. Naast de proof-of-concept zal er ook een literatuurstudie en een volledige documentatie zijn van het proces.

%---------- Stand van zaken ---------------------------------------------------

\section{Literatuurstudie}%
\label{sec:state-of-the-art}

\subsection{Geschiedenis gamificatie}

Een van de eerste en meest succesvolle projecten waarbij gamificatie werd toegepast was door het bedrijf Sperry en Hutchinson Co. opgericht in 1896 te Alabama. Hun idee om mensen te motiveren met spel elementen hebben ze geïmplementeerd door het invoeren van stempels waarmee klanten dan producten konden kopen. \autocite{Christians2018}. \\

Daarna duurde het een hele periode alvorens er een evolutie kwam in dit gebied. Gamificatie zoals we het vandaag de dag kennen werd geboren in het jaar 2003. Nick Pelling kwam namelijk af met het idee om speelvolle elementen toe te voegen aan bepaalde producten. Zijn idee was sterk en de term was geboren maar zijn bedrijf ging uiteindelijk ten onder. \autocite{Khaitova2021} \\

De volgende evolutie kwam er in 2010 toen Jane McGonigal opperde dat gamers de wereld konden veranderen als ze maar de juiste richting in gedirigeerd werden. Door de positieve ontvangst van het publiek leidde deze uitspraak tot de eerste Gamification Summit een jaar later. \autocite{Christians2018} \\

Het begrip zelf evolueerde na deze veranderingen niet meer maar werd wel steeds vaker geïmplementeerd. 

\subsection{Gamificatie en sport}

Gamificatie is al reeds aanwezig in verschillende sport applicaties. Bij meesten, zoals bijvoorbeeld bij Strava, is er het gebruik van verschillende vriendelijke competities waar een top 3 of top 10 word gevisualiseerd. De keuze voor deze manier van aanpak kan wel negatieve gevolgen hebben omdat het er voor kan zorgen dat mensen gedemotiveerd raken doordat ze buiten die top 10 of zelfs top 20 vallen. \autocite{Alen2020} Het is dus belangrijk om oftewel te zorgen dat de speler ten alle tijden zich betrokken voelt tot het spel element ofdat er een gezamenlijk doel is om naar samen te werken.

\subsection{Strava API}

De reden waarom gekozen is voor Strava is vanwege het feit dat Strava niet platform gebonden is doordat het een applicatie is die niet gebonden is aan externe bepaalde particuliere hardware. \autocite{David2023} Hierdoor is er geen drempel voor nieuwe sporters of uitsluiting op basis van materiaal dat de iemand al dan niet bezit. 

Een hoeksteen waar de applicatie zwaar op zal steunen is de API aangeleverd door de sport app Strava. Strava zelf is een applicatie gebruikt door sportievelingen om informatie van hun sportprestaties bij te houden. \\

De API aangeboden door dit bedrijf is gratis maar als ontwikkelaar heb je wel een account nodig. Data kan niet opgehaald worden in grote getallen van eender wie zijn account maar elke gebruiker moet zich aanmelden met zijn of haar eigen account via 0Auth 2.0. Er zijn ook enkele limitaties aan de API namelijk request limits van 200 per 15 minuten en een maximum van 2000 requests per dag. \autocite{Strava2023}

\section{Methodologie}%
\label{sec:methodologie}

\subsection{Deelnemers}

De deelnemers van deze studie zullen bestaan uit verschillende jongeren tussen 16-21 jaar als testers voor de applicatie. Dit omdat zij de uiteindelijke eindgebruikers zullen zijn van de applicatie moest deze in circulatie gezet worden.

\subsection{Ontwerp en procedure}

Voor dit onderzoek is het de bedoeling om uiteindelijk tot een proof-of-concept te geraken dat kan afgeleverd worden. Hiervoor werden de volgende stappen onderverdeeld in bepaalde fasen:

\begin{enumerate}
    \item \hyperref[subsub:beschrijving]{Beschrijving casus}
    \item \hyperref[subsub:uitwerking]{Uitwerken POC}
    \item \hyperref[subsub:testfase]{Testen van de applicatie}
    \item \hyperref[subsub:conclusies]{Conclusies}
\end{enumerate}

\subsection{Beschrijving casus}
\label{subsub:beschrijving}

Gedurende deze eerste fase is het belangrijk om informatie te verzamelen omtrent hoe het gamificatie aspect verwerkt zal worden in de applicatie. Ook een achtergrondstudie naar het grondprobleem dat zal worden aangepakt en wat de beste manieren zijn om dit probleem aan te pakken vormen een groot aspect van deze fase. \\

\subsection{Uitwerken Proof-of-Concept}
\label{subsub:uitwerking}

Deze tweede fase zal ook degene worden waar de meeste tijd in zal kruipen. Dit deel bestaat er namelijk uit om de volledige POC uit te werken. Deze applicatie zal als front-end functioneel uitgebouwd worden met React en als back-end zal het rechtstreeks steunen op de Strava API. 

\subsection{Testen van de applicatie}
\label{subsub:testfase}

Gedurende deze fase zal de applicatie aan automatische testen onderworpen worden in verband met beveiliging en veiligheid. Daarna kan de applicatie open gesteld worden voor een selecte groep product testers die de kans krijgen het product uit te testen en daarna via een formulier de kans krijgen om hun mening en ervaringen op een gestructureerde manier door te geven zodat deze dan kunnen verwerkt worden. De data die hier centraal staat is de gebruikstijden van de applicatie over verloop van tijd als ook het aantal afgelegde kilometers per test persoon over het verloop van tijd. Dit zal dan uiteindelijk aangevuld worden met een persoonlijke enquete waar de ervaringen van de testers in zal opgenomen worden. 

\subsection{Conclusies}
\label{subsub:conclusies}

Als laatste fase is het belangrijk om te reflecteren op het gehele proces en de data te verwerken die verzameld is doorheen de fase ervoor. Hieruit kan dan een conclusie opgemaakt worden. Dit kan dan dienen als advies voor anderen naar de toekomst toe. 



%---------- Verwachte resultaten ----------------------------------------------
\section{Verwacht resultaat, conclusie}%
\label{sec:verwachte_resultaten}

Uit de verwachte resultaten kan opgemaakt worden of een gamificatie van sportprestaties een nuttige manier is om verder te gaan om zo meer mensen te motiveren om regelmatiger te sporten. Verwacht is dat door de stimulatie aan hand van medailles en levels dat meer mensen gemotiveerder zullen zijn door een gevoel te hebben dat ze ergens naartoe werken dat reeël en duidelijk afgebakend is.
